\usepackage{babel}
\selectlanguage{german}

\usepackage[scaled]{helvet}
\usepackage[T1]{fontenc}

\renewcommand*{\familydefault}{\sfdefault}
\usepackage{ngerman}

\usepackage{color}
\usepackage{xcolor, soul}
\sethlcolor{cyan}
\usepackage{tabularx}
\usepackage[table]{xcolor}
\usepackage{array}

\usepackage{graphicx}
\usepackage{wrapfig}

\usepackage{setspace}
\usepackage{geometry}

\usepackage{tikz}
\usetikzlibrary{shapes.geometric, arrows, arrows.meta, positioning, calc, trees}

%Tikz Flowchart
\tikzstyle{startstop} = [rectangle, rounded corners, minimum width=3cm, minimum height=1cm,text centered, draw=black, fill=red!30]
\tikzstyle{io} = [trapezium, trapezium left angle=70, trapezium right angle=110, minimum width=3cm, minimum height=1cm, text centered, draw=black, fill=blue!30]
\tikzstyle{process} = [rectangle, minimum width=3cm, minimum height=1cm, text centered, draw=black, fill=orange!30]
\tikzstyle{decision} = [diamond, minimum width=3cm, minimum height=1cm, text centered, draw=black, fill=green!30]
\tikzstyle{arrow} = [thick,->,>=stealth]

%Tikz ERD

\tikzstyle{entity} = [rectangle, minimum width=3cm, minimum height=1cm,text centered, draw=black, fill=red!30]
\tikzstyle{relationship} = [diamond, aspect=2, minimum width=3cm, minimum height=1cm, text centered, draw=black, fill=blue!30]
\tikzstyle{attribute} = [ellipse, minimum width=3cm, minimum height=1cm, text centered, draw=black, fill=orange!30]

\usepackage{url}
\usepackage{pdfpages}

\usepackage[
bookmarks,
bookmarksnumbered,
bookmarksopen=true,
bookmarksopenlevel=1,
colorlinks=true,
anchorcolor=blue,
citecolor=blue,
filecolor=blue,
menucolor=blue,
urlcolor=blue,
%linkcolor=black,
%anchorcolor=black,
%citecolor=black,
%filecolor=black,
%menucolor=black, 
%urlcolor=black,
pdftex,
plainpages=false,
pdfpagelabels=true, 
hypertexnames=false,
linkcolor=black,
linktoc=all,
]{hyperref}

\hypersetup{
	pdftitle={Pruefungsvorbereitung},
	pdfauthor={Karina},
	pdfcreator={Karina},
	pdfsubject={Pruefungsvorbereitung},
	pdfkeywords={Pruefungsvorbereitung},
}

\usepackage{multicol}
\usepackage{enumitem}