\section{Netzwerktechnik}
\label{sec:Netzwerktechnik}

\todo{Fehlt}

\begin{itemize}
	\item Adressierung
	\begin{itemize}
		\item IPv4/IPv6, MAC, ARP
	\end{itemize}
	\item Routing, Switching
	\item DNS, DHCP
	\item TCP/UDP
	\item HTTPS, TLS/SSL, IPsec
	\begin{itemize}
		\item Hash, Signatur, Zertifikat, Certificate Authority
	\end{itemize}
	\item Verschlüsselung (pre-shared key, RADIUS …)
	\item LAN/WAN/MAN/GAN
	\item Strukturierte Verkabelung
	\begin{itemize}
		\item primäre/sekundäre/tertiäre Verkabelung
		\item Kabeltypen (Twisted Pair, LWL)
	\end{itemize}
	\item VLAN
	\item Sicherheitskonzepte und -risiken: WEP, WPA
	\item Netzwerktopologien
	\item Netzwerkplan
	\item VPN
	\begin{itemize}
		\item Funktionsweise und Vorteile von VPN beschreiben
		\item VPN-Modelle
		\item Tunneling
	\end{itemize}
	\item Serverarten: Mailserver, Webserver, Groupware, Datenbanken, Proxy
	\item Sicherstellung des Betriebs
	\begin{itemize}
		\item Elektrotechnisch (USV)
		\item Hardwaretechnisch (Redundanzen), RAID
		\item Softwaretechnisch (Back-ups…)
	\end{itemize}
	\item Firewall
	\item Portsecurity, Port-Forwarding
\end{itemize}

\subsection{Addressierung}

\paragraph{IPv4/IPv6}

IPv4 (Internet Protocol Version 4) und IPv6 (Internet Protocol Version 6) sind Protokolle, die zur Adressierung von Geräten in einem Netzwerk verwendet werden. IPv4 verwendet 32-Bit-Adressen und ist das am häufigsten verwendete Protokoll im Internet. Es hat jedoch eine begrenzte Anzahl von verfügbaren Adressen, was mit dem Wachstum des Internets zu Engpässen geführt hat. IPv6 hingegen verwendet 128-Bit-Adressen und wurde entwickelt, um dieses Problem zu lösen und zukünftiges Wachstum zu ermöglichen.

\paragraph{MAC}

MAC (Media Access Control) bezieht sich auf die physische Adresse eines Netzwerkgeräts, die auch als Hardwareadresse bekannt ist. Diese Adresse wird in der Netzwerkkarte eines Geräts eingebettet und dient zur eindeutigen Identifizierung innerhalb eines lokalen Netzwerks. MAC-Adressen sind in der Regel in Form von sechs Doppelpunkten getrennten Hexadezimalzahlen dargestellt und werden von Netzwerkprotokollen wie Ethernet verwendet.

\paragraph{ARP}

ARP (Address Resolution Protocol) ist ein Netzwerkprotokoll, das verwendet wird, um die IP-Adresse eines Netzwerkgeräts in die entsprechende MAC-Adresse umzuwandeln. Wenn ein Gerät in einem lokalen Netzwerk Daten an ein anderes Gerät senden möchte, benötigt es die MAC-Adresse des Zielgeräts. ARP ermöglicht es, diese Zuordnung von IP-Adressen zu MAC-Adressen dynamisch zu ermitteln und zu speichern, indem es ARP-Anfragen sendet und ARP-Antworten empfängt. So kann effizienter Datenverkehr im Netzwerk stattfinden.

\subsection{Routing, Switching}

\paragraph{Routing}

Routing bezieht sich auf den Prozess der Weiterleitung von Datenpaketen zwischen verschiedenen Netzwerken oder Subnetzen, um den besten Weg für den Datenverkehr zu finden. Dabei werden Routing-Algorithmen verwendet, die basierend auf verschiedenen Kriterien wie Kosten, Latenzzeit oder Bandbreite entscheiden, welcher Pfad für die Übertragung von Daten am besten geeignet ist. Router sind Geräte, die für das Routing zuständig sind und Datenpakete entsprechend weiterleiten.

\paragraph{Switching}

Switching ist der Prozess des Weiterleitens von Datenpaketen innerhalb eines lokalen Netzwerks. Im Gegensatz zu Routern, die Daten zwischen verschiedenen Netzwerken weiterleiten, arbeiten Switches auf der Ebene des lokalen Netzwerks und verbinden verschiedene Geräte innerhalb desselben Netzwerks miteinander. Switches verwenden MAC-Adressen, um Datenpakete an das richtige Zielgerät innerhalb des Netzwerks zu senden, was zu einer effizienten Datenübertragung und -kommunikation in lokalen Netzwerken führt.

\paragraph{Test}TEST

\clearpage