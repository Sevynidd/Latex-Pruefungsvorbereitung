\section{Netzwerktechnik}
\label{sec:Netzwerktechnik}

\subsection{ISO/OSI Modell}

\paragraph{ISO/OSI Modell} Das ISO/OSI-Modell (International Organization for Standardization/Open Systems Interconnection Model) ist ein Referenzmodell, das die Grundlagen für die Kommunikation in Rechnernetzen definiert. Es unterteilt den Kommunikationsprozess in sieben Schichten, wobei jede Schicht bestimmte Funktionen und Dienste für die Datenübertragung bereitstellt. Diese Schichten sind:
\begin{enumerate}
	\item \textbf{Physikalische Schicht (Physical Layer):} Diese Schicht ist für die physische Übertragung von Datenbits über das Kommunikationsmedium verantwortlich. Sie beschäftigt sich mit Aspekten wie Signalisierung, Übertragungsraten und elektrischen Eigenschaften der Verbindungen.\\
	\textbf{Protokolle:} Token Ring
	\item \textbf{Sicherungsschicht (Data Link Layer):} Die Sicherungsschicht ist für die fehlerfreie Übertragung von Datenrahmen zwischen benachbarten Netzwerkknoten verantwortlich. Sie befasst sich mit Themen wie Rahmenbildung, Adressierung, Fehlererkennung und -korrektur sowie Zugriffskontrolle.\\
	\textbf{Protokolle:} MAC, Ethernet, Token Ring
	\item \textbf{Netzwerkschicht (Network Layer):} Diese Schicht ist für die Weiterleitung von Datenpaketen von einem Quell- zum Zielknoten über ein Netzwerk verantwortlich. Sie bietet Funktionen wie Routing, Adressierung und Vermittlung von Daten zwischen verschiedenen Subnetzen.\\
	\textbf{Protokolle:} IP, IPsec, ICMP
	\item \textbf{Transportschicht (Transport Layer):} Die Transportschicht ist für die zuverlässige Übertragung von Daten zwischen Endpunkten (z.B. Hosts) im Netzwerk verantwortlich. Sie stellt sicher, dass Datenpakete korrekt und in der richtigen Reihenfolge übertragen werden und bietet Mechanismen zur Fehlerbehebung und Flusskontrolle.\\
	\textbf{Protokolle:} TCP, UDP
	\item \textbf{Sitzungsschicht (Session Layer):} Die Sitzungsschicht ist für die Verwaltung von Sitzungen oder Verbindungen zwischen Anwendungen auf verschiedenen Netzwerkknoten verantwortlich. Sie ermöglicht die Einrichtung, Aufrechterhaltung und Beendigung von Kommunikationssitzungen sowie die Synchronisierung und Wiederherstellung von Datenübertragungen.\\
	\textbf{Protokolle:} DHCP, DNS, FTP, HTTP, HTTPS, LDAP, SMTP
	\item \textbf{Darstellungsschicht (Presentation Layer):} Diese Schicht ist für die Umwandlung von Daten in ein für die Übertragung geeignetes Format verantwortlich und umgekehrt. Sie kümmert sich um Aspekte wie Datenkompression, Verschlüsselung, Formatierung und Kodierung, um sicherzustellen, dass Daten zwischen verschiedenen Systemen verstanden werden können.\\
	\textbf{Protokolle:} DHCP, DNS, FTP, HTTP, HTTPS, LDAP, SMTP
	\item \textbf{Anwendungsschicht (Application Layer):} Die Anwendungsschicht ist die oberste Schicht des ISO/OSI-Modells und stellt Dienste und Schnittstellen bereit, die Anwendungen den Zugriff auf das Netzwerk ermöglichen. Sie umfasst Protokolle und Dienste für Anwendungen wie E-Mail, Dateiübertragung, Remotezugriff und Webbrowser.\\
	\textbf{Protokolle:} DHCP, DNS, FTP, HTTP, HTTPS, LDAP, SMTP
\end{enumerate}

\subsection{Addressierung}

\paragraph{IPv4/IPv6}

IPv4 (Internet Protocol Version 4) und IPv6 (Internet Protocol Version 6) sind Protokolle, die zur Adressierung von Geräten in einem Netzwerk verwendet werden. IPv4 verwendet 32-Bit-Adressen und ist das am häufigsten verwendete Protokoll im Internet. Es hat jedoch eine begrenzte Anzahl von verfügbaren Adressen, was mit dem Wachstum des Internets zu Engpässen geführt hat. IPv6 hingegen verwendet 128-Bit-Adressen und wurde entwickelt, um dieses Problem zu lösen und zukünftiges Wachstum zu ermöglichen.

\paragraph{MAC}

MAC (Media Access Control) bezieht sich auf die physische Adresse eines Netzwerkgeräts, die auch als Hardwareadresse bekannt ist. Diese Adresse wird in der Netzwerkkarte eines Geräts eingebettet und dient zur eindeutigen Identifizierung innerhalb eines lokalen Netzwerks. MAC-Adressen sind in der Regel in Form von sechs Doppelpunkten getrennten Hexadezimalzahlen dargestellt und werden von Netzwerkprotokollen wie Ethernet verwendet.

\paragraph{ARP}

ARP (Address Resolution Protocol) ist ein Netzwerkprotokoll, das verwendet wird, um die IP-Adresse eines Netzwerkgeräts in die entsprechende MAC-Adresse umzuwandeln. Wenn ein Gerät in einem lokalen Netzwerk Daten an ein anderes Gerät senden möchte, benötigt es die MAC-Adresse des Zielgeräts. ARP ermöglicht es, diese Zuordnung von IP-Adressen zu MAC-Adressen dynamisch zu ermitteln und zu speichern, indem es ARP-Anfragen sendet und ARP-Antworten empfängt. So kann effizienter Datenverkehr im Netzwerk stattfinden.

\subsection{Routing, Switching}

\paragraph{Routing}

Routing bezieht sich auf den Prozess der Weiterleitung von Datenpaketen zwischen verschiedenen Netzwerken oder Subnetzen, um den besten Weg für den Datenverkehr zu finden. Dabei werden Routing-Algorithmen verwendet, die basierend auf verschiedenen Kriterien wie Kosten, Latenzzeit oder Bandbreite entscheiden, welcher Pfad für die Übertragung von Daten am besten geeignet ist. Router sind Geräte, die für das Routing zuständig sind und Datenpakete entsprechend weiterleiten.

\paragraph{Switching}

Switching ist der Prozess des Weiterleitens von Datenpaketen innerhalb eines lokalen Netzwerks. Im Gegensatz zu Routern, die Daten zwischen verschiedenen Netzwerken weiterleiten, arbeiten Switches auf der Ebene des lokalen Netzwerks und verbinden verschiedene Geräte innerhalb desselben Netzwerks miteinander. Switches verwenden MAC-Adressen, um Datenpakete an das richtige Zielgerät innerhalb des Netzwerks zu senden, was zu einer effizienten Datenübertragung und -kommunikation in lokalen Netzwerken führt.

\subsection{DNS, DHCP}

\paragraph{DNS}

DNS (Domain Name System) ist ein hierarchisches und verteiltes System zur Auflösung von Domainnamen in IP-Adressen und umgekehrt. Es ermöglicht die Verwendung von leicht zu merkenden Domainnamen, wie z.B. 'example.com', anstelle von IP-Adressen, wie z.B. '192.0.2.1', um auf Websites, Server und andere Ressourcen im Internet zuzugreifen. DNS funktioniert, indem es Anfragen von Clients entgegennimmt und diese Anfragen an DNS-Server weiterleitet, die die entsprechenden IP-Adressen zurückgeben.

\paragraph{DHCP}

DHCP (Dynamic Host Configuration Protocol) ist ein Netzwerkprotokoll, das automatisch IP-Adressen, Subnetzmasken, Standard-Gateways und andere Netzwerkkonfigurationen an Geräte in einem Netzwerk verteilt. DHCP ermöglicht es, dass Geräte, die sich in einem Netzwerk anmelden, automatisch eine IP-Adresse und andere Netzwerkkonfigurationen erhalten, ohne dass der Netzwerkadministrator manuell jeden Client konfigurieren muss. Dies erleichtert die Verwaltung von IP-Adressen in einem Netzwerk und reduziert potenzielle Konflikte.

\subsection{TCP/UDP}

\paragraph{TCP}

TCP (Transmission Control Protocol) ist ein zuverlässiges, verbindungsorientiertes Protokoll, das für die Übertragung von Daten über Netzwerke verwendet wird. Es gewährleistet die zuverlässige und geordnete Zustellung von Datenpaketen, indem es Bestätigungen für den Erhalt von Datenpaketen verwendet und bei Bedarf fehlende Pakete erneut sendet. TCP wird häufig für Anwendungen verwendet, bei denen eine fehlerfreie und vollständige Übertragung von Daten erforderlich ist, wie z.B. beim Abrufen von Webseiten, dem Senden von E-Mails oder dem Herunterladen von Dateien.

\paragraph{UDP}

UDP (User Datagram Protocol) ist ein unzuverlässiges, verbindungsloses Protokoll, das für die Übertragung von Daten über Netzwerke verwendet wird. Im Gegensatz zu TCP bietet UDP keine Garantie für die zuverlässige Zustellung von Daten oder die Beibehaltung der Reihenfolge. UDP wird häufig für Anwendungen verwendet, bei denen eine niedrige Latenz und eine schnelle Datenübertragung wichtiger sind als die Zuverlässigkeit, wie z.B. bei Voice-over-IP (VoIP), Videostreaming und Online-Spielen.

\subsection{HTTPS, TLS/SSL, IPsec}

\paragraph{HTTPS}

HTTPS (Hypertext Transfer Protocol Secure) ist ein Protokoll zur sicheren Übertragung von Daten über das Internet. Es basiert auf dem HTTP-Protokoll, verwendet jedoch zusätzliche Verschlüsselungsschichten, um die Vertraulichkeit und Integrität der übertragenen Daten zu gewährleisten. HTTPS wird häufig für die sichere Übertragung von sensiblen Daten wie Login-Informationen, Kreditkartendaten und persönlichen Informationen auf Websites verwendet. Es verwendet SSL (Secure Sockets Layer) oder TLS (Transport Layer Security) zur Verschlüsselung der Datenübertragung.

\paragraph{TLS/SSL}

TLS (Transport Layer Security) und sein Vorgänger SSL (Secure Sockets Layer) sind Verschlüsselungsprotokolle, die zur Sicherung der Kommunikation über das Internet verwendet werden. Sie bieten Sicherheitsebenen, indem sie Verschlüsselung, Authentifizierung und Integritätsschutz für die übertragenen Daten bieten. TLS wird häufig in Kombination mit anderen Protokollen wie HTTPS (sicheres HTTP), SMTPS (sicheres SMTP) und FTPS (sicheres FTP) eingesetzt, um eine sichere Kommunikation zwischen Clients und Servern zu gewährleisten.

\paragraph{IPsec}

IPsec (Internet Protocol Security) ist ein Protokollsuite, die zur Sicherung der Kommunikation auf Netzwerkebene verwendet wird. Es bietet Vertraulichkeit, Integrität und Authentizität für die übertragenen IP-Pakete, indem es Verschlüsselung und Authentifizierung verwendet. IPsec wird häufig in Virtual Private Networks (VPNs) eingesetzt, um eine sichere Kommunikation über unsichere Netzwerke wie das Internet zu ermöglichen. Es kann in zwei Modi betrieben werden: Transportmodus, der nur die Nutzdaten verschlüsselt, und Tunnelmodus, der das gesamte IP-Paket einschließlich Header verschlüsselt.

\paragraph{Hash}

Ein Hash ist eine kryptografische Funktion, die eine Eingabe (Nachricht, Daten) in eine feste Länge von Daten umwandelt, die als Hash-Wert bezeichnet wird. Der Hash-Wert ist im Allgemeinen eine eindeutige, nicht umkehrbare Darstellung der Eingabe. Hash-Funktionen werden häufig verwendet, um die Integrität von Daten zu überprüfen, da eine geringfügige Änderung an den Eingabedaten zu einem völlig anderen Hash-Wert führt. Sie werden auch in Passwort-Hashes, digitalen Signaturen und anderen kryptografischen Anwendungen verwendet.

\paragraph{Signatur}

Eine Signatur ist ein kryptografisches Verfahren, das verwendet wird, um die Authentizität, Integrität und Nichtabstreitbarkeit von Daten zu gewährleisten. Sie wird normalerweise mit einem privaten Schlüssel erstellt und kann mit einem öffentlichen Schlüssel überprüft werden. Durch Signieren von Daten mit einem privaten Schlüssel können Sender ihre Identität bestätigen und sicherstellen, dass die Daten nicht manipuliert wurden. Die Empfänger können dann die Signatur mit dem öffentlichen Schlüssel des Senders überprüfen, um die Echtheit der Daten zu bestätigen.

\paragraph{Zertifikate}

Ein Zertifikat ist eine digitale Bescheinigung, die die Identität einer Entität (Person, Organisation, Website) authentifiziert und von einer vertrauenswürdigen Zertifizierungsstelle (Certificate Authority, CA) signiert ist. Zertifikate enthalten öffentliche Schlüssel und andere Informationen über die identifizierte Entität. Sie werden häufig in der SSL/TLS-Verschlüsselung verwendet, um die Identität von Websites zu bestätigen und die sichere Kommunikation zu gewährleisten.

\paragraph{Certificate Authority}

Eine Certificate Authority (CA) ist eine vertrauenswürdige Organisation, die digitale Zertifikate ausstellt und signiert, um die Identität von Entitäten im Internet zu authentifizieren. CAs sind für die Verifizierung der Identität von Antragstellern, die Ausstellung von Zertifikaten und die Verwaltung von Zertifikatswiderrufung verantwortlich. Zu den bekannten CAs gehören Unternehmen wie Let's Encrypt, VeriSign und Comodo, die von den gängigen Webbrowsern und Betriebssystemen als vertrauenswürdig eingestuft werden.

\subsection{Verschlüsselung}

\paragraph{Symmetrische Verschlüsselung}

Die symmetrische Verschlüsselung ist ein Verfahren, bei dem derselbe Schlüssel zum Verschlüsseln und Entschlüsseln von Daten verwendet wird. Sowohl der Sender als auch der Empfänger müssen denselben geheimen Schlüssel kennen, um die Kommunikation zu verschlüsseln und zu entschlüsseln. Symmetrische Verschlüsselungsalgorithmen wie AES (Advanced Encryption Standard) sind schnell und effizient, werden jedoch oft durch das Schlüsselverwaltungsproblem eingeschränkt, da der Schlüssel sicher zwischen den Parteien ausgetauscht werden muss.

\paragraph{Asymmetrische Verschlüsselung}

Die asymmetrische Verschlüsselung, auch Public-Key-Verschlüsselung genannt, verwendet zwei unterschiedliche Schlüssel: einen öffentlichen Schlüssel und einen privaten Schlüssel. Der öffentliche Schlüssel wird zum Verschlüsseln von Daten verwendet und kann frei verteilt werden, während der private Schlüssel zum Entschlüsseln der Daten verwendet wird und geheim gehalten werden muss. Dieses Verfahren ermöglicht sichere Kommunikation, da der private Schlüssel nicht offenbart wird. Asymmetrische Verschlüsselung wird häufig für digitale Signaturen, sichere E-Mail-Kommunikation und die Erstellung von SSL/TLS-Zertifikaten verwendet.

\paragraph{RADIUS}

RADIUS (Remote Authentication Dial-In User Service) ist ein Netzwerkprotokoll, das zur Authentifizierung, Autorisierung und Buchhaltung von Benutzern in einem Netzwerk verwendet wird, insbesondere in drahtlosen Netzwerken und virtuellen privaten Netzwerken (VPNs). RADIUS ermöglicht es Netzwerkgeräten wie WLAN-Zugriffspunkten, Switches und VPN-Konzentratoren, Benutzerinformationen über ein Netzwerk zu überprüfen und zu autorisieren, bevor sie den Netzwerkzugriff gewähren. Es bietet auch Funktionen zur Buchführung und Protokollierung von Benutzeraktivitäten im Netzwerk.

\paragraph{Pre-shared key}

Ein Pre-shared key (PSK) ist ein gemeinsamer geheimer Schlüssel, der zwischen zwei oder mehr Parteien vor der Kommunikation vereinbart wird. PSKs werden häufig in verschiedenen Sicherheitsprotokollen und -mechanismen verwendet, um die Authentizität und Vertraulichkeit der übertragenen Daten zu gewährleisten. Beispiele für den Einsatz von PSKs sind in WLAN-Netzwerken, VPN-Verbindungen und drahtlosen Sensornetzwerken, wo sie zur Authentifizierung und Verschlüsselung des Datenverkehrs verwendet werden. Im Gegensatz zu öffentlichen Schlüsselverfahren erfordern PSKs keine Infrastruktur für Zertifikate oder öffentliche Schlüsselverwaltung und sind daher oft einfacher einzurichten und zu verwalten.

\subsection{Netzwerktypen}

\paragraph{LAN}

Ein Local Area Network (LAN) ist ein Netzwerk, das Geräte in einem begrenzten geografischen Bereich wie einem Gebäude, einer Wohnung oder einem Campus miteinander verbindet. LANs ermöglichen den Austausch von Daten und Ressourcen wie Dateien, Druckern und Anwendungen zwischen den verbundenen Geräten. Sie werden häufig in Unternehmen, Bildungseinrichtungen und Privathaushalten eingesetzt und verwenden in der Regel Ethernet- oder WLAN-Technologien für die Verbindung der Geräte.

\paragraph{WAN}

Ein Wide Area Network (WAN) ist ein Netzwerk, das Geräte über große geografische Entfernungen miteinander verbindet, oft über öffentliche Telekommunikationsnetze wie das Internet oder private dedizierte Leitungen. WANs ermöglichen die Kommunikation zwischen entfernten Standorten, Niederlassungen und Rechenzentren und unterstützen verschiedene Dienste wie E-Mail, Webzugriff, VoIP und Videokonferenzen. WANs können sowohl öffentliche als auch private Verbindungen verwenden und erfordern in der Regel spezielle Netzwerkgeräte wie Router und Switches.

\paragraph{MAN}

Ein Metropolitan Area Network (MAN) ist ein Netzwerk, das mehrere LANs in einem geografisch begrenzten städtischen Gebiet miteinander verbindet. MANs bieten eine höhere Bandbreite und Reichweite als LANs und ermöglichen die Kommunikation zwischen verschiedenen Standorten innerhalb einer Stadt oder einer städtischen Region. Sie werden häufig von Unternehmen, Regierungsbehörden und Serviceanbietern eingesetzt, um standortübergreifende Dienste wie VoIP, Datenübertragung und Videoüberwachung bereitzustellen.

\paragraph{GAN}

Ein Global Area Network (GAN) ist ein Netzwerk, das mehrere WANs miteinander verbindet und eine weltweite Kommunikation ermöglicht. GANs nutzen verschiedene Technologien wie Satellitenverbindungen, Unterseekabel und drahtlose Netzwerke, um eine nahtlose Konnektivität über große geografische Entfernungen hinweg zu gewährleisten. Sie werden häufig von multinationalen Unternehmen, Telekommunikationsunternehmen und Regierungsbehörden eingesetzt, um weltweite Kommunikationsdienste anzubieten und globale Geschäftsaktivitäten zu unterstützen.

\paragraph{SAN}

Ein Storage Area Network (SAN) ist ein dediziertes Netzwerk, das die Verbindung von Speichergeräten wie Festplatten, Solid-State-Laufwerken (SSDs) und Speicherarrays mit Servern ermöglicht. SANs werden häufig in Rechenzentren und Unternehmensumgebungen eingesetzt, um eine zentrale und hochverfügbare Speicherlösung bereitzustellen. Sie ermöglichen es mehreren Servern, gleichzeitig auf gemeinsame Speicherressourcen zuzugreifen, was eine flexible und skalierbare Speicherinfrastruktur bietet. SANs verwenden oft spezielle Netzwerkprotokolle wie Fibre Channel oder iSCSI zur Datenübertragung zwischen den Servern und Speichergeräten.

\paragraph{PAN}

Die Abkürzung 'PAN' steht für 'Personal Area Network' (Persönliches Netzwerk). Ein PAN ist ein Netzwerk, das sich in einem begrenzten Bereich um eine Person herum erstreckt und verschiedene Geräte miteinander verbindet, die sich in unmittelbarer Nähe befinden. Typische Geräte in einem PAN sind Computer, Smartphones, Tablets, Drucker, Headsets und andere persönliche elektronische Geräte.

\subsection{Strukturierte Verkabelung}

\paragraph{Primäre/Sekundäre/Tertiäre Verkabelung}

Bei der strukturierten Verkabelung werden die Kabeltypen je nach ihrer Verwendung in primäre, sekundäre und tertiäre Verkabelung unterteilt.

- Die primäre Verkabelung umfasst die Hauptverkabelung, die das gesamte Gebäude oder die gesamte Anlage abdeckt. Sie verbindet die Telekommunikationsräume, Verteiler und Etagenverteiler miteinander.

- Die sekundäre Verkabelung umfasst die Verkabelung innerhalb einzelner Etagen oder Abteilungen. Sie verbindet die Verteiler mit den Anschlussdosen und sorgt für die horizontale Verbindung zu den Endgeräten.

- Die tertiäre Verkabelung umfasst die Verkabelung innerhalb eines Raumes oder einer Arbeitsplatzgruppe. Sie umfasst die Anschlussdosen, Patchfelder und Patchkabel, die die Endgeräte mit dem Netzwerk verbinden.

\paragraph{Kabeltypen}

\subparagraph{Twisted Pair}

Twisted Pair ist ein Kabeltyp, der aus mehreren Kupferdrahtpaaren besteht, die miteinander verdrillt sind. Diese Verdrillung reduziert elektromagnetische Störungen und verbessert die Signalqualität. Twisted-Pair-Kabel werden häufig für die Verkabelung von Ethernet-Netzwerken, Telefonleitungen und anderen Datenübertragungsanwendungen verwendet. Es gibt verschiedene Kategorien von Twisted-Pair-Kabeln, wie z.B. Cat5e, Cat6 und Cat6a, die unterschiedliche Bandbreiten und Übertragungsgeschwindigkeiten bieten.

\subparagraph{LWL}

LWL (Lichtwellenleiter) oder Glasfaserkabel bestehen aus dünnen Glas- oder Kunststofffasern, die Lichtsignale zur Übertragung von Daten verwenden. Sie bieten eine hohe Bandbreite, geringe Dämpfung und Immunität gegen elektromagnetische Interferenzen, was sie ideal für die Übertragung großer Datenmengen über große Entfernungen macht. LWL-Kabel werden häufig in Backbone-Netzwerken, Telekommunikationsnetzen und Rechenzentren eingesetzt, wo hohe Übertragungsraten und Zuverlässigkeit erforderlich sind.

\subsection{VLAN}

\paragraph{VLAN}

Ein Virtual Local Area Network (VLAN) ist ein logisches Netzwerk, das aus Geräten in verschiedenen physischen Netzwerken besteht, die auf der Grundlage von Portnummern, MAC-Adressen, Protokollen oder anderen Kriterien gruppiert sind. VLANs ermöglichen es, ein physisches Netzwerk in mehrere virtuelle Netzwerke zu unterteilen, wodurch die Sicherheit, Skalierbarkeit und Verwaltbarkeit verbessert werden. Sie können verwendet werden, um Benutzer in verschiedenen Abteilungen, virtuelle Maschinen, Gäste und andere Gruppen zu isolieren und den Datenverkehr zwischen den VLANs zu steuern.

\subsection{Sicherheitskonzepte und -risiken: WEB, WPA}

\paragraph{WEB}

Wired Equivalent Privacy (WEP) ist ein veraltetes Sicherheitsprotokoll, das zur Verschlüsselung von Daten in drahtlosen Netzwerken verwendet wurde. WEP verwendet symmetrische Verschlüsselung mit einem gemeinsamen Schlüssel, um die Vertraulichkeit des Datenverkehrs zu gewährleisten. Es war jedoch anfällig für verschiedene Sicherheitslücken und Schwachstellen, die es Angreifern ermöglichten, den WEP-Schlüssel zu knacken und auf den Netzwerkverkehr zuzugreifen. Aufgrund seiner Schwächen wurde WEP durch sicherere Protokolle wie WPA und WPA2 ersetzt.

\paragraph{WPA}

Wi-Fi Protected Access (WPA) ist ein Sicherheitsprotokoll, das entwickelt wurde, um die Sicherheit von drahtlosen Netzwerken zu verbessern und die Schwächen von WEP zu überwinden. WPA verwendet verbesserte Verschlüsselungsalgorithmen wie TKIP (Temporal Key Integrity Protocol) und AES (Advanced Encryption Standard), um die Vertraulichkeit und Integrität des Datenverkehrs zu gewährleisten. Es bietet auch Funktionen wie Pre-shared Key (PSK), 802.1X-Authentifizierung und dynamische Schlüsselwechsel, um die Sicherheit des drahtlosen Netzwerks weiter zu erhöhen.

\subsection{Netzwerktopologien}

\paragraph{Netzwerktopologien}

Eine Netzwerktopologie beschreibt die physische oder logische Anordnung von Geräten, Kabeln und anderen Komponenten in einem Netzwerk. Es gibt verschiedene Arten von Netzwerktopologien, die je nach den Anforderungen und den spezifischen Einsatzszenarien eines Netzwerks eingesetzt werden können.

\begin{itemize}
	\item Die Bus-Topologie besteht aus einem einzigen Kommunikationskanal, der von allen Geräten im Netzwerk gemeinsam genutzt wird. Alle Geräte sind direkt mit diesem Bus verbunden, was zu einer einfachen und kostengünstigen Verkabelung führt. Ein Ausfall eines Geräts kann jedoch das gesamte Netzwerk beeinträchtigen.
	\item Die Stern-Topologie besteht aus einem zentralen Knoten, der alle anderen Geräte im Netzwerk verbindet. Alle Datenverbindungen laufen über diesen zentralen Knoten, was eine einfache Erweiterung und Verwaltung ermöglicht. Ein Ausfall des zentralen Knotens kann jedoch das gesamte Netzwerk lahmlegen.
	\item Die Ring-Topologie besteht aus einer Reihe von Geräten, die in einer geschlossenen Schleife verbunden sind. Daten werden in einer Richtung von einem Gerät zum nächsten übertragen, bis sie ihr Ziel erreichen. Diese Topologie bietet eine gleichmäßige Lastverteilung und hohe Zuverlässigkeit, ist jedoch anfällig für Ausfälle, da der Ausfall eines einzigen Geräts den gesamten Ring unterbrechen kann.
	\item Die Stern-basierte Ring-Topologie kombiniert die Vorteile der Stern- und Ring-Topologien, indem sie eine Sternstruktur mit einem Ringverbindungsnetzwerk innerhalb des Sterns verwendet. Dies bietet die Vorteile einer einfachen Erweiterung und Verwaltung sowie einer gleichmäßigen Lastverteilung und hohen Zuverlässigkeit.
	\item Die Mesh-Topologie besteht aus einer Reihe von Geräten, die direkt miteinander verbunden sind und mehrere Pfade zur Kommunikation bieten. Diese Topologie bietet eine hohe Redundanz und Ausfallsicherheit, ist jedoch aufgrund der großen Anzahl von Verbindungen zwischen den Geräten teuer und schwer zu verwalten.
\end{itemize}

\subsection{Netzwerkplan}

\paragraph{Netzwerkplan}

Ein Netzwerkplan ist eine schematische Darstellung eines Netzwerks, die die physische und logische Struktur sowie die Konnektivität der verschiedenen Netzwerkkomponenten zeigt. Ein Netzwerkplan kann verschiedene Elemente enthalten, wie z.B. Router, Switches, Server, Arbeitsstationen, Verbindungslinien, IP-Adressen und VLANs. Er dient als Referenzdokument für Netzwerkadministratoren und IT-Techniker, um das Netzwerkdesign zu visualisieren, Änderungen zu planen und Probleme zu diagnostizieren.

\subsection{VPN}

\paragraph{Funktionsweise von VPN}

Ein Virtual Private Network (VPN) ist eine Technologie, die es Benutzern ermöglicht, eine sichere Verbindung zu einem privaten Netzwerk über ein öffentliches Netzwerk wie das Internet herzustellen. VPNs verwenden Verschlüsselung und andere Sicherheitsmechanismen, um den Datenverkehr zwischen dem Benutzer und dem privaten Netzwerk zu schützen und die Privatsphäre und Sicherheit zu gewährleisten. VPNs können verwendet werden, um sichere Remote-Zugriffe auf Unternehmensnetzwerke, den Schutz der Privatsphäre bei der Internetnutzung und die Umgehung geografischer Einschränkungen zu ermöglichen.

\paragraph{VPN-Modelle}

Es gibt verschiedene VPN-Modelle, die je nach den Anforderungen und dem Einsatzzweck eines Netzwerks eingesetzt werden können:

\begin{itemize}
	\item Remote Access VPN: Ermöglicht autorisierten Benutzern den sicheren Zugriff auf das Unternehmensnetzwerk von entfernten Standorten aus über das Internet.
	\item Site-to-Site VPN: Verbindet zwei oder mehr physische Standorte oder Netzwerke miteinander und ermöglicht den sicheren Austausch von Daten über das Internet.
	\item Extranet VPN: Ermöglicht autorisierten externen Benutzern oder Partnern den sicheren Zugriff auf bestimmte Ressourcen oder Dienste im Unternehmensnetzwerk.
	\item Intranet VPN: Bietet sicheren Zugriff auf interne Ressourcen und Dienste innerhalb eines Unternehmensnetzwerks für autorisierte Benutzer, unabhängig von ihrem Standort.
\end{itemize}

\paragraph{Tunneling}

Tunneling ist ein Prozess, bei dem Datenpakete in einem anderen Netzwerkprotokoll eingekapselt werden, um sie sicher über ein unsicheres Netzwerk zu übertragen. Bei der VPN-Kommunikation wird Tunneling verwendet, um den privaten Datenverkehr über das öffentliche Internet zu transportieren, ohne dass Dritte den Inhalt der Daten sehen oder manipulieren können. Die Daten werden verschlüsselt und in Pakete eines anderen Protokolls eingekapselt, bevor sie über das Internet gesendet werden. Am Zielort werden die Pakete wieder entkapselt und entschlüsselt, um die ursprünglichen Daten wiederherzustellen.

\subsection{Serverarten}

\paragraph{Mailserver}

Ein Mailserver ist ein spezieller Server, der E-Mails empfängt, speichert, weiterleitet und zustellt. Er fungiert als zentrale Komponente eines E-Mail-Systems und ermöglicht es Benutzern, E-Mails zu senden und zu empfangen. Mailserver verwenden verschiedene Protokolle wie SMTP (Simple Mail Transfer Protocol), POP3 (Post Office Protocol Version 3) und IMAP (Internet Message Access Protocol) für die Kommunikation zwischen den Clients und dem Server.

\paragraph{Webserver}

Ein Webserver ist ein Server, der Webinhalte wie Webseiten, Bilder, Videos und andere Dateien über das World Wide Web bereitstellt. Er empfängt HTTP-Anfragen von Webbrowsern und sendet die angeforderten Dateien als HTTP-Antworten zurück. Webserver hosten Websites und Webanwendungen und verwenden häufig Serversoftware wie Apache, Nginx, Microsoft IIS und andere.

\paragraph{Groupware}

Groupware bezeichnet Softwareanwendungen, die die Zusammenarbeit und Kommunikation in einer Gruppe oder Organisation erleichtern. Ein Groupware-Server ermöglicht es Benutzern, gemeinsam an Projekten zu arbeiten, Termine zu planen, Dokumente zu teilen, E-Mails zu senden und andere kollaborative Aktivitäten durchzuführen. Beispiele für Groupware-Server sind Microsoft Exchange Server, IBM Notes/Domino und Open-Xchange.

\paragraph{Datenbanken}

Ein Datenbankserver ist ein spezieller Server, der die Speicherung, Verwaltung und Abfrage von Daten in einer Datenbank ermöglicht. Er fungiert als zentrale Schnittstelle für Benutzer und Anwendungen, um auf die Datenbank zuzugreifen und mit ihr zu interagieren. Datenbankserver verwenden SQL (Structured Query Language) oder andere Abfragesprachen, um Daten zu manipulieren und abzurufen. SQL-Datenbanken, wie MySQL, PostgreSQL, Oracle Database und Microsoft SQL Server, folgen einem relationalen Datenbankmodell und speichern Daten in Tabellen, die durch Beziehungen miteinander verbunden sind. Im Gegensatz dazu verwenden NoSQL-Datenbanken, wie MongoDB, Couchbase und Cassandra, ein nicht-relationales Datenbankmodell, das keine festen Tabellenstrukturen erfordert. NoSQL-Datenbanken sind flexibler und skalierbarer für die Speicherung unstrukturierter Daten, wie beispielsweise in Big-Data-Anwendungen, aber SQL-Datenbanken sind oft besser geeignet für komplexe Abfragen und Transaktionen in traditionellen Unternehmensanwendungen.

\paragraph{Proxy}

Ein Proxyserver ist ein Server, der als Vermittler zwischen Clientgeräten und anderen Servern im Internet fungiert. Er empfängt Anfragen von Clientgeräten wie Webbrowsern und leitet sie an andere Server weiter. Proxyserver können verschiedene Funktionen erfüllen, darunter die Verbesserung der Sicherheit und Privatsphäre, die Optimierung der Netzwerk- und Webbeschleunigung sowie die Filterung und Blockierung von unerwünschtem Datenverkehr. Sie werden häufig in Unternehmensnetzwerken, Schulen und öffentlichen WLANs eingesetzt.

\subsection{Sicherstellung des Betriebs}

\subparagraph{USV}

Eine USV (Unterbrechungsfreie Stromversorgung) ist eine elektrotechnische Vorrichtung, die dazu dient, elektronische Geräte vor Stromausfällen und Spannungsschwankungen zu schützen. Sie besteht aus einem Akku, der bei einem Stromausfall automatisch einspringt und den angeschlossenen Geräten kontinuierlich Strom liefert, bis die Stromversorgung wiederhergestellt ist oder die Geräte sicher heruntergefahren werden können. USVs sind besonders wichtig für Server und andere kritische IT-Systeme, um Datenverluste und Betriebsunterbrechungen zu vermeiden.

\subparagraph{RAID}

RAID (Redundant Array of Independent Disks) ist eine hardwaretechnische Methode zur Erhöhung der Datensicherheit und -verfügbarkeit durch die redundante Speicherung von Daten über mehrere Festplatten. Es gibt verschiedene RAID-Level, die unterschiedliche Methoden der Datenspiegelung, Parität und Striping verwenden, um die Leistung, Zuverlässigkeit und Kapazität der Speicherlösung zu optimieren. RAID ermöglicht es, Daten auch bei Ausfall einer Festplatte weiterhin verfügbar zu halten und die Ausfalltoleranz des Systems zu verbessern.

\subparagraph{Back-ups}

Back-ups sind eine softwaretechnische Maßnahme zur Sicherung und Wiederherstellung von Daten im Falle von Datenverlust, Hardwarefehlern, menschlichen Fehlern oder Katastrophen. Durch regelmäßige Back-ups werden Kopien der Daten erstellt und an einem sicheren Ort gespeichert, um im Bedarfsfall die Wiederherstellung der Daten zu ermöglichen. Back-ups können auf lokalen Speichermedien wie Festplatten oder Bandlaufwerken sowie in der Cloud gespeichert werden. Sie sind ein wichtiger Bestandteil der Datensicherungsstrategie eines Unternehmens und dienen dem Schutz vor Datenverlust und Geschäftsunterbrechungen.

\subsection{Firewall}

\paragraph{Firewall}

Eine Firewall ist eine Sicherheitsvorrichtung, die dazu dient, ein Netzwerk vor unerwünschtem Zugriff, Datenverkehr und potenziell schädlichen Angriffen aus dem Internet oder anderen Netzwerken zu schützen. Sie überwacht den Datenverkehr zwischen einem internen Netzwerk und externen Netzwerken und filtert oder blockiert Datenpakete basierend auf vordefinierten Sicherheitsregeln. Firewalls können auf Netzwerkgeräten wie Routern, Switches oder dedizierten Firewall-Appliances implementiert werden und bieten Funktionen wie Paketfilterung, Zustandsinspektion, Anwendungsfilterung und Virtual Private Network (VPN)-Unterstützung.

\subsection{Portsecurity, Port-Forwarding}

\paragraph{Portsecurity}

Portsecurity ist eine Sicherheitsfunktion, die dazu dient, unautorisierten Zugriff auf Netzwerkgeräte wie Switches zu verhindern, indem sie den Zugriff auf bestimmte Netzwerkports basierend auf der MAC-Adresse des angeschlossenen Geräts beschränkt. Durch die Konfiguration von Portsecurity können Administratoren die Anzahl der Geräte begrenzen, die an einen Switchport angeschlossen werden können, und unbekannte oder nicht autorisierte Geräte blockieren oder alarmieren, die versuchen, auf das Netzwerk zuzugreifen.

\paragraph{Port-Forwarding}

Port-Forwarding ist eine Netzwerkkonfigurationstechnik, die verwendet wird, um den Datenverkehr von einem bestimmten Port auf einem Netzwerkgerät an einen anderen Port auf einem anderen Gerät in einem lokalen Netzwerk oder im Internet weiterzuleiten. Es ermöglicht die Weiterleitung eingehender Netzwerkanfragen von einem externen Netzwerk an bestimmte Dienste oder Anwendungen, die auf internen Servern oder Endgeräten ausgeführt werden. Port-Forwarding wird häufig in Heimnetzwerken, Unternehmen und Rechenzentren eingesetzt, um den Zugriff auf interne Ressourcen wie Webserver, FTP-Server, Spiele-Server und Remote-Desktop-Verbindungen zu ermöglichen.

\clearpage