IHK Belegsatz \cite{IHKBelegsatz}

\begin{center}
	\setlength\arrayrulewidth{1pt}
	\begin{tabular}{|p{0.45\textwidth} | p{0.45\textwidth}|}
		\hline
		\textbf{Syntax} & \textbf{Beschreibung} \\
		\hline
		\rowcolor{tableLightGray} \textit{Tabelle} &  \\
		\hline
		\begin{tabular}[t]{l}
			\textbf{CREATE TABLE} Tabellenname(\\ Spaltenname <DATENTYP>,\\ Primärschlüssel,\\ Fremdschlüssel)
		\end{tabular} & Erzeugt eine neue leere Tabelle mit der beschriebenen Struktur \\
		\hline
		\begin{tabular}[t]{l}
			\textbf{ALTER TABLE} Tabellenname\\ \hspace{.5em} \textbf{ADD COLUMN} Spaltenname Datentyp\\ \hspace{.5em} \textbf{DROP COLUMN} Spaltenname Datentyp\\ \\ \textbf{ADD FOREIGN KEY}(Spaltenname)\\ \hspace{1em} \textbf{REFERENCES} Tabellenname(\\ \hspace{8em} Primärschlüssel) \\
		\end{tabular} & 
		\begin{tabular}[t]{l}
			Änderungen an einer Tabelle:\\
			\hspace{1em}Hinzufügen einer Spalte\\
			\hspace{1em}Entfernen einer Spalte\\\\
			\hspace{1em}Definiert eine Spalte als Fremdschlüssel
		\end{tabular}\\
		\hline
		\textbf{CHARACTER} & Textdatentyp \\
		\hline
		\textbf{DECIMAL} & Numerischer Datentyp (Festkommazahl)\\
		\hline
		\textbf{DOUBLE} & Numerischer Datentyp (Doppelte Präzision)\\
		\hline
		\textbf{INTEGER} & Numerischer Datentyp (Ganzzahl)\\
		\hline
		\textbf{DATE} & Datum (Format DD.MM.YYYY)\\
		\hline
		\textbf{PRIMARY KEY} (Spaltenname) & Erstellung eines Primärschlüssels \\
		\hline
		\begin{tabular}[t]{l}
			\textbf{FOREIGN KEY} (Spaltenname)\\
			\hspace{1em} \textbf{REFERENCES} Tabellenname(\\
			\hspace{2em}Primärschlüsselspaltenname)
		\end{tabular} & Erstellung einer Fremdschlüssel-Beziehung\\
		\hline
		\textbf{DROP TABLE} Tabellenname & Löscht eine Tabelle\\
		\hline
		\rowcolor{tableLightGray} \textit{Befehle, Klauseln, Attribute} & \\
		\hline
		\textbf{SELECT *} | Spaltenname1 [, Spaltenname2, ...] & Wählt die Spalten einer oder mehrerer Tabellen, deren Inhalte in die Liste aufgenommen werden sollen; alle Spalten (*) oder die namentlich aufgeführten\\
		\hline
		\textbf{FROM} & Name der Tabelle oder Namen der Tabellen, aus denen die Daten der Ausgabe stammen sollen\\
		\hline
		\begin{tabular}[t]{l}
			\textbf{SELECT ...}\\
			\textbf{FROM ...}\\
			\hspace{1em}\textbf{(SELECT ...}\\
			\hspace{1em}\textbf{FROM ...}\\
			\hspace{1em}\textbf{WHERE ...) AS} tbl\\
			\textbf{WHERE ...}
		\end{tabular} & Unterabfrage (subquery), die in eine äußere Abfrage eingebettet ist. Das Ergebnis der Unterabfrage wird wie eine Tabelle - hier mit Namen 'tbl' - behandelt. \\
		\hline
	\end{tabular}
\end{center}

\begin{center}
	\setlength\arrayrulewidth{1pt}
	\begin{tabular}{|p{0.45\textwidth} | p{0.45\textwidth}|}
		\hline
		SELECT \textbf{DISTINCT} & Eliminiert Redundanzen, die in einer Tabelle auftreten können. Werte werden jeweils nur einmal angezeigt.\\
		\hline
		\textbf{JOIN / INNER JOIN} & Liefert nur die Datensätze zweier Tabellen, die gleiche Datenwerte enthalten\\
		\hline
		\textbf{LEFT JOIN / LEFT OUTER JOIN} & Liefert von der erstgenannten (linken) Tabelle alle Datensätze und von der zweiten Tabelle jene, deren Datenwerte mit denen der ersten Tabelle übereinstimmen\\
		\hline
		\textbf{RIGHT JOIN / RIGHT OUTER JOIN} & Liefert von der zweiten (rechten) Tabelle alle Datensätze und von der ersten Tabelle jene, deren Datenwerte mit denen der zweiten Tabelle übereinstimmen\\
		\hline
		\textbf{WHERE} & Bedingung, nach der Datensätze ausgewählt werden sollen\\
		\hline
		\begin{tabular}[t]{l}
			\textbf{WHERE EXISTS} (subquery)\\
			\textbf{WHERE NOT EXISTS} (subquery)
		\end{tabular} & Die Bedingung EXISTS prüft, ob die Suchbedingung einer Unterabfrage mindestens eine Zeile zurückliefert. NO EXISTS negiert die Bedingung.\\
		\hline
		\begin{tabular}[t]{l}
			\textbf{WHERE ... IN} (subquery)\\\\
			\textbf{WHERE NOT ... IN} (subquery)
		\end{tabular} & 
		\begin{tabular}[t]{l}
			Der Wert des Datenfelds ist in der \\
			ausgewählten Menge vorhanden.\\
			Der Wert des Datenfelds ist in der \\
			ausgewählten Menge nicht vorhanden.
		\end{tabular}\\
		\hline
		\textbf{GROUP BY} Spaltenname1 [, Spaltenname2, ...] & Gruppierung (Aggregation) nach Inhalt des genannten Feldes\\
		\hline
		\textbf{ORDER BY} Spaltenname1 [, Spaltenname2, ...] \textbf{ASC | DESC} & Sortierung nach Inhalt des genannten Feldes oder der genannten Felder ASC: aufsteigend; DESC: absteigend\\
		\hline
		\rowcolor{tableLightGray} \textit{Datenmanipulation} & \\
		\hline
		\textbf{DELETE FROM} Tabellenname & Löschen von Datensätzen in der genannten Tabelle\\
		\hline
		\textbf{UPDATE} Tabellenname \textbf{SET} & Aktualisiert Daten in Feldern einer Tabelle\\
		\hline
		\begin{tabular}[t]{l}
			\textbf{INSERT INTO} Tabellenname\\
			\hspace{8em}[(spalte1, spalte2, ...)]\\
			\hspace{1em}\textbf{VALUES} (Wert Spalte 1 \\
			\hspace{8em}[, Wert Spalte 2, ...])\\
			oder\\
			\hspace{1em}\textbf{SELECT ... FROM ... WHERE}
		\end{tabular} & Fügt Datensätze in die genannte Tabelle ein, die entweder mit festen Werten belegt oder Ergebnis eines SELECT-Befehls sind\\
		\hline
		\rowcolor{tableLightGray} \textit{Berechtigungen kontrollieren} & \\
		\hline
		\begin{tabular}[t]{l}
				\textbf{CREATE} Benutzer | Rolle \textbf{IDENTIFIED BY}\\ 'Passwort'
		\end{tabular} & 
		\begin{tabular}[t]{l}
			Erzeugt einen neuen Benutzer oder eine \\ 
			neue Rolle mit einem Passwort
		\end{tabular}\\
		\hline		
	\end{tabular}
\end{center}

\begin{center}
	\setlength\arrayrulewidth{1pt}
	\begin{tabular}{|p{0.45\textwidth} | p{0.45\textwidth}|}
		\hline
		\begin{tabular}[t]{l}
			\textbf{GRANT} Recht | Rolle \textbf{ON} *.* | Datenbank.* | \\
			Datenbank.Objekt\\
			\textbf{TO} Benutzer | Rolle [WITH GRANT OPTION]
		\end{tabular} & 
		\begin{tabular}[t]{l}
			Weist einem Benutzer oder einer Rolle ein \\
			Recht auf ein
			bestimmtes Datenbank-Objekt\\
			Weist einem Benutzer eine Rolle zu
		\end{tabular}\\
		\hline
		\begin{tabular}[t]{l}
			\textbf{REVOKE} Recht | Rolle \textbf{ON} *.* | Datenbank.* |\\
			Datenbank.Objekt\\
			\textbf{FROM} Benutzer | Rolle
		\end{tabular} & 
		\begin{tabular}[t]{l}
			Entzieht einem Benutzer oder einer Rolle ein \\
			Recht auf ein
			bestimmtes Datenbank-Objekt\\
			Entzieht einem Benutzer eine Rolle
		\end{tabular}\\
		\hline
		\rowcolor{tableLightGray} \textit{Aggregatfunktionen} & \\
		\hline
		\textbf{AVG} (Spaltenname) & Ermittelt das arithmetische Mittel aller Werte im angegebenen Feld\\
		\hline
		\textbf{COUNT} (Spaltenname | *) & Ermittelt die Anzahl der Datensätze mit Nicht-NULL-Werten im angegebenen Feld oder alle Datensätze der Tabelle (dann mit Operator *)\\
		\hline
		\textbf{SUM} (Spaltenname | Formel) & Ermittelt die Summe aller Werte im angegebenen Feld oder der Formelergebnisse\\
		\hline
		\textbf{MIN} (Spaltenname | Formel) & Ermittelt den kleinsten aller Werte im angegebenen Feld\\
		\hline
		\textbf{MAX} (Spaltenname | Formel) & Ermittelt den größten aller Werte im angegebenen Feld\\
		\hline
		\rowcolor{tableLightGray} \textit{Funktionen} & \\
		\hline
		\textbf{LEFT} (Zeichenkette, Anzahlzeichen) & Liefert \textit{Anzahlzeichen} der Zeichenkette von links.\\
		\hline
		\textbf{RIGHT} (Zeichenkette, Anzahlzeichen) & Liefert \textit{Anzahlzeichen} der Zeichenkette von rechts.\\
		\hline
		\textbf{CURRENT} & Liefert das aktuelle Datum mit der aktuellen Uhrzeit\\
		\hline
		\textbf{CONVERT(time,}[DatumZeit]) & Liefert die Uhrzeit aus einer DatumZeit-Angabe\\
		\hline
		\textbf{DATE} (Wert) & Wandelt einen Wert in ein Datum um\\
		\hline
		\textbf{DAY} (Datum) & Liefert den Tag des Monats aus dem angegebenen Datum\\
		\hline
		\textbf{MONTH} (Datum) & Liefert den Monat aus dem angegebenen Datum\\
		\hline
		\textbf{TODAY} & Liefert das aktuelle Datum\\
		\hline
		\textbf{WEEKDAY} (Datum) & Liefert den Tag der Woche aus dem angegebenen Datum\\
		\hline
		\textbf{YEAR} (Datum) & Liefert das Jahr aus dem angegebenen Datum\\
		\hline
		\textbf{DATEADD} (Datumsteil, Intervall, Datum) & Fügt einem Datum ein Intervall (ausgefrückt in den unter Datumsteil angegebenen Einheiten) hinzu\\
		\hline
		\textbf{DATEDIFF} (Datumsteil, Anfangsdatum, Enddatum) Datumsteile: \textbf{DAY, MONTH, YEAR} & Liefert Enddatum-Startdatum (ausgedrückt in den unter Datumsteil angegebenen Einheiten)\\
		\hline
	\end{tabular}
\end{center}

\begin{center}
	\setlength\arrayrulewidth{1pt}
	\begin{tabular}{|p{0.45\textwidth} | p{0.45\textwidth}|}
		\hline
		\rowcolor{tableLightGray} \textit{Operatoren} & \\
		\hline
		\textbf{AND} & Logisches UND\\
		\hline
		\textbf{LIKE} & Überprüfung von Text auf Gleichheit wenn Platzhalter ('regular expressions') eingesetzt werden.\\
		\hline
		\textbf{NOT} & Logische Negation\\
		\hline
		\textbf{OR} & Logisches ODER\\
		\hline
		\textbf{IS NULL} & Überprüfung auf NULL\\
		\hline
		\textbf{=} & Test auf Gleichheit\\
		\hline
		\textbf{>, >=, <, <=, <>} & Test auf Ungleichheit\\
		\hline
		\textbf{*} & Multiplikation\\
		\hline
		\textbf{/} & Division\\
		\hline
		\textbf{+} & Addition, positives Vorzeichen\\
		\hline
		\textbf{-} & Subtraktion, negatives Vorzeigen\\
		\hline
	\end{tabular}
\end{center}