\subsection{SQL}
\label{sec:SQL}

Alle SQL-Komponenten für Abfragen siehe \referenz{sec:Datenbanken}.

\subsubsection{Projektion vs. Selektion}
\label{sec:ProjektionSelektion}

Quelle: Tino Hempel \cite{projektionSelektion}

\paragraph{Selektion} Bei der Selektion werden \hl{Zeilen aus einer Tabelle} ausgewählt, die bestimmten Eigenschaften genügen.

Aus der Tabelle Schüler sollen alle Zeilen selektiert werden, in denen der Name "Müller" steht. 
Die Selektion hat also die Form: $S_{Name} = _{'Mueller'}(Schueler)$

\vspace{1em}

\begin{minipage}{.45\textwidth}
	\begin{center}
		Schüler \\
		\vspace{1em}
		\bgroup
		\setlength{\tabcolsep}{1em}
		\def\arraystretch{1.5}
		\begin{tabular}{|c|l|l|}
			\hline
			\rowcolor{tableLightGray}\underline{SNr} & Vorname & Name \\
			\hline
			4711 & Paul & Müller \\
			\hline
			0815 & Erich & Schmidt \\
			\hline
			7472 & Sven & Lehmann \\
			\hline
			1234 & Olaf & Müller \\
			\hline
			2313 & Jürgen & Paulsen \\
			\hline
		\end{tabular}
		\egroup
	\end{center}
\end{minipage}
\hfill
\begin{minipage}{.45\textwidth}
	\begin{center}
		$S_{Name} = _{'Mueller'}(Schueler)$ \\
		\vspace{1em}
		\bgroup
		\setlength{\tabcolsep}{1em}
		\def\arraystretch{1.5}
		\begin{tabular}{|c|l|l|}
			\hline
			\rowcolor{tableLightGray}\underline{SNr} & Vorname & Name \\
			\hline
			12 & Paul & Müller \\
			\hline
			308 & Olaf & Müller \\
			\hline
		\end{tabular}
		\egroup
	\end{center}
\end{minipage}

\paragraph{Projektion} Bei der Projektion werden \hl{Spalten aus einer Tabelle} ausgewählt, die bestimmten Eigenschaften genügen.

Aus der Tabelle Schüler sollen alle Spalten mit dem Attribut 'Name' projiziert werden. 
Die Projektion hat also die Form: $P_{Name}(Schueler)$

\vspace{1em}

\begin{minipage}{.45\textwidth}
	\begin{center}
		Schüler \\
		\vspace{1em}
		\bgroup
		\setlength{\tabcolsep}{1em}
		\def\arraystretch{1.5}
		\begin{tabular}{|c|l|l|}
			\hline
			\rowcolor{tableLightGray}\underline{SNr} & Vorname & Name \\
			\hline
			4711 & Paul & Müller \\
			\hline
			0815 & Erich & Schmidt \\
			\hline
			7472 & Sven & Lehmann \\
			\hline
			1234 & Olaf & Müller \\
			\hline
			2313 & Jürgen & Paulsen \\
			\hline
		\end{tabular}
		\egroup
	\end{center}
\end{minipage}
\hfill
\begin{minipage}{.45\textwidth}
	\begin{center}
		$P_{Name}(Schueler)$ \\
		\vspace{1em}
		\bgroup
		\setlength{\tabcolsep}{1em}
		\def\arraystretch{1.5}
		\begin{tabular}{|l|}
			\hline
			\rowcolor{tableLightGray} Name \\
			\hline
			Müller \\
			\hline
			Schmidt \\
			\hline
			Lehmann \\
			\hline
			Müller \\
			\hline
			Paulsen \\
			\hline
		\end{tabular}
		\egroup
	\end{center}
\end{minipage}


\subsubsection{DDL, DML \& DCL}
\label{sec:DDLDMLDCL}

Quelle: GeeksforGeeks \cite{DDLDMLDCL}

\begin{enumerate}
	\item DDL - Data Definition Language
	\item DML - Data Manipulation Language
	\item DQL - Data Query Language
	\item DCL - Data Control Language
\end{enumerate}

\begin{center}
	\begin{tikzpicture}[
	item/.style={rectangle,draw,fill=blue!30, inner sep = 10},
	subitem/.style={rectangle,draw,fill=blue!20, inner sep = 8},
	grandchild/.style={grow=down,xshift=1em,anchor=west,
		edge from parent path={(\tikzparentnode.south) |- (\tikzchildnode.west)}},
	first/.style={level distance=6ex},
	second/.style={level distance=12ex},
	third/.style={level distance=18ex},
	forth/.style={level distance=24ex},
	level 1/.style={sibling distance=10em}]
	% Parents
	\coordinate
	node[item,anchor=east]{SQL Commands}
	child[grow=down,level distance=2.5ex]
	[edge from parent fork down]
	% Children and grandchildren
	child{node[item] {DDL}
		child[grandchild,first] {node[subitem]{CREATE}}
		child[grandchild,second] {node[subitem]{DROP}}
		child[grandchild,third] {node[subitem] {ALTER}}
		child[grandchild, forth] {node[subitem] {TRUNCATE}}}
	child{node[item] {DML}
		child[grandchild,first] {node[subitem]{INSERT}}
		child[grandchild,second] {node[subitem]{UPDATE}}
		child[grandchild,third] {node[subitem]{DELETE}}}
	child {node[item]{DQL}
		child[grandchild,first] {node[subitem]{SELECT}}}
	child {node[item]{DCL}
		child[grandchild,first] {node[subitem]{GRANT}}
		child[grandchild,second] {node[subitem]{REVOKE}}};
\end{tikzpicture}
\end{center}

\subsubsection{CRUD}
\label{sec:CRUD}

Quelle: sqlshack.com \cite{crud}

\paragraph{Create} refers to CREATE aka add, insert. In this operation, it is expected to insert a new record using the SQL insert statement. SQL uses \hl{INSERT INTO} statement to create new records within the table.

\vspace{1em}

\begin{lstlisting}[language=SQL]
	INSERT INTO <tablename> (column1,column2,...)
	
	VALUES (value1,value2,...)
\end{lstlisting}

\begin{lstlisting}[language=SQL]
	INSERT INTO dbo.Demo
	VALUES
	(1, 
	'Prashanth'
	);
\end{lstlisting}

\begin{lstlisting}[language=SQL]
	INSERT INTO <tablename> (column1,column2,...) 
	VALUES(value1,value2,...),( value1,value2,...), (value1,value2,...)...
\end{lstlisting}



\subsubsection{Subquerys}
\label{sec:Subquery}

\subsubsection{Aggregatfunktionen}
\label{sec:Aggregatfunktionen}

\subsubsection{Schnitt-, Vereinigungs- und Differenzmenge}
\label{sec:SchnittVereinigungsDifferenzmenge}

\subsubsection{SQL Injection}
\label{sec:SQLInjection}

