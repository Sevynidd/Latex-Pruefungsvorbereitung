\subsection{SQL}
\label{sec:SQL}

Alle SQL-Komponenten für Abfragen siehe \referenz{sec:Datenbanken}.

\subsubsection{Projektion vs. Selektion}
\label{sec:ProjektionSelektion}

Quelle: Tino Hempel \cite{projektionSelektion}

\paragraph{Selektion} Bei der Selektion werden \hl{Zeilen aus einer Tabelle} ausgewählt, die bestimmten Eigenschaften genügen.

Aus der Tabelle Schüler sollen alle Zeilen selektiert werden, in denen der Name "Müller" steht. 
Die Selektion hat also die Form: $S_{Name} = _{'Mueller'}(Schueler)$

\vspace{1em}

\begin{minipage}{.45\textwidth}
	\begin{center}
		Schüler \\
		\vspace{1em}
		\bgroup
		\setlength{\tabcolsep}{1em}
		\def\arraystretch{1.5}
		\begin{tabular}{|c|l|l|}
			\hline
			\rowcolor{tableLightGray}\underline{SNr} & Vorname & Name \\
			\hline
			4711 & Paul & Müller \\
			\hline
			0815 & Erich & Schmidt \\
			\hline
			7472 & Sven & Lehmann \\
			\hline
			1234 & Olaf & Müller \\
			\hline
			2313 & Jürgen & Paulsen \\
			\hline
		\end{tabular}
		\egroup
	\end{center}
\end{minipage}
\hfill
\begin{minipage}{.45\textwidth}
	\begin{center}
		$S_{Name} = _{'Mueller'}(Schueler)$ \\
		\vspace{1em}
		\bgroup
		\setlength{\tabcolsep}{1em}
		\def\arraystretch{1.5}
		\begin{tabular}{|c|l|l|}
			\hline
			\rowcolor{tableLightGray}\underline{SNr} & Vorname & Name \\
			\hline
			12 & Paul & Müller \\
			\hline
			308 & Olaf & Müller \\
			\hline
		\end{tabular}
		\egroup
	\end{center}
\end{minipage}

\paragraph{Projektion} Bei der Projektion werden \hl{Spalten aus einer Tabelle} ausgewählt, die bestimmten Eigenschaften genügen.

Aus der Tabelle Schüler sollen alle Spalten mit dem Attribut 'Name' projiziert werden. 
Die Projektion hat also die Form: $P_{Name}(Schueler)$

\vspace{1em}

\begin{minipage}{.45\textwidth}
	\begin{center}
		Schüler \\
		\vspace{1em}
		\bgroup
		\setlength{\tabcolsep}{1em}
		\def\arraystretch{1.5}
		\begin{tabular}{|c|l|l|}
			\hline
			\rowcolor{tableLightGray}\underline{SNr} & Vorname & Name \\
			\hline
			4711 & Paul & Müller \\
			\hline
			0815 & Erich & Schmidt \\
			\hline
			7472 & Sven & Lehmann \\
			\hline
			1234 & Olaf & Müller \\
			\hline
			2313 & Jürgen & Paulsen \\
			\hline
		\end{tabular}
		\egroup
	\end{center}
\end{minipage}
\hfill
\begin{minipage}{.45\textwidth}
	\begin{center}
		$P_{Name}(Schueler)$ \\
		\vspace{1em}
		\bgroup
		\setlength{\tabcolsep}{1em}
		\def\arraystretch{1.5}
		\begin{tabular}{|l|}
			\hline
			\rowcolor{tableLightGray} Name \\
			\hline
			Müller \\
			\hline
			Schmidt \\
			\hline
			Lehmann \\
			\hline
			Müller \\
			\hline
			Paulsen \\
			\hline
		\end{tabular}
		\egroup
	\end{center}
\end{minipage}


\subsubsection{DDL, DML \& DCL}
\label{sec:DDLDMLDCL}

Quelle: GeeksforGeeks \cite{DDLDMLDCL}

\begin{enumerate}
	\item DDL - Data Definition Language
	\item DQL - Data Query Language
	\item DML - Data Manipulation Language
	\item DCL - Data Control Language
	\item TCL - Transaction Control Language
\end{enumerate}

\paragraph{DDL} besteht aus den SQL-Befehlen, die zur Definition des Datenbankschemas verwendet werden können. Sie befasst sich lediglich mit Beschreibungen des Datenbankschemas und wird zur Erstellung und Änderung der Struktur von Datenbankobjekten in der Datenbank verwendet. DDL ist ein Satz von SQL-Befehlen, die zum Erstellen, Ändern und Löschen von Datenbankstrukturen, aber nicht von Daten verwendet werden. Diese Befehle werden normalerweise nicht von einem normalen Benutzer verwendet, der über eine Anwendung auf die Datenbank zugreifen sollte.

\begin{itemize}
	\item CREATE
	\item DROP
	\item ALTER
	\item TRUNCATE
	\begin{itemize}
		\item Dieser Befehl wird verwendet, um alle Datensätze aus einer Tabelle zu entfernen, einschließlich aller für die Datensätze zugewiesenen Speicherplätze
	\end{itemize}
	\item COMMENT
	\item RENAME
	\begin{itemize}
		\item Dient zum Umbenennen eines in der Datenbank vorhandenen Objekts
	\end{itemize}
\end{itemize}

\paragraph{DQL} 
