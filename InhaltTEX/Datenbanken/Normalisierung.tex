\subsection{Normalisierung}
\label{sec:Normalisierung}

Quelle: DatabaseCamp \cite{normalisierung}

\paragraph{1. Normalform} Eine Relation liegt in der ersten Normalform vor, wenn alle Attributwerte \hl{atomar} vorliegen.

Das bedeutet, dass jedes Datenfeld lediglich einen Wert enthalten darf. Außerdem sollte sichergestellt sein, dass jede Spalte nur Werte desselben Datentyps (Numerisch, Text, etc.) enthält. Folgende Beispiele müssten entsprechend verändert werden, damit eine Datenbank in der 1. Normalform vorhanden ist:

\begin{multicols}{2}
	\begin{itemize}
		\item Adresse: “Hauptstraße 1, 12345 Berlin” 
		\begin{itemize}
			\item Straße: “Hauptstraße” 
			\item Hausnummer: “1”
			\item PLZ: “12345”
			\item Ort: “Berlin”
		\end{itemize}
		\columnbreak
		\item Rechnungsbetrag: “128,45 €”
		\begin{itemize}
			\item Betrag: “128,45”
			\item Währung: “€”
			\vfill
		\end{itemize}
	\end{itemize}
\end{multicols}

\paragraph{2. Normalform} Eine Relation liegt in der zweiten Normalform vor, wenn sie in der ersten Normalform vorliegt und alle \hl{Nichtschlüsselattribute voll funktional vom gesamten Primärschlüssel abhängig} sind.

Der Primärschlüssel bezeichnet ein Attribut, das zur eindeutigen Identifikation einer Datenbankzeile verwendet werden kann. Dazu zählen beispielsweise die Rechnungsnummer zur Identifikation einer Rechnung oder die Ausweisnummer zur Identifikation einer Person.

Konkret bedeutet dies in der Anwendung, dass alle Merkmale ausgelagert werden müssen, die nicht ausschließlich vom Primärschlüssel abhängig sind. In der Praxis führt dies dann oft zu einem sogenannten Sternschema.



\paragraph{3. Normalform} Eine Relation liegt in der dritten Normalform vor, wenn sie in der ersten und zweiten Normalform vorliegt und \hl{keine transitiven Abhängigkeiten} bestehen.

\subsubsection{Anomalien}

\paragraph{Einfügeanomalie}

\paragraph{Änderungsanomalie}

\paragraph{Löschanomalie}