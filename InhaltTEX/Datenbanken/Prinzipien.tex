\subsection{Prinzipien}
\label{sec:DBPrinzipien}

\paragraph{ACID}
ACID steht für die vier englischen Einzelbegriffe Atomicity, Consistency, Isolation und Durability und ist eine gängige Abkürzung der Informationstechnik. Im Deutschen lauten die vier Begriffe Atomarität, Konsistenz, Isolation und Dauerhaftigkeit. Oft wird im deutschsprachigen Raum das Akronym AKID verwendet. Das ACID-Prinzip stellt Regeln auf, wie mit Transaktionen in Datenbankmanagementsystemen zu verfahren ist, um für verlässliche, konsistente Daten und Systeme zu sorgen.

Eine Transaktion besteht aus einer Folge verschiedener Vorgänge, die diese ACID-Regeln einzuhalten haben. Geprägt wurde das Akronym ACID bereits im Jahr 1983 von den beiden Informatikern Andreas Reuter und Theo Härder. In Normen wie ISO/IEC 10.026-1:1992 oder ISO/IEC 10.026-1:1998 Abschnitt 4 ist das ACID-Prinzip beschrieben. Zur Umsetzung des ACID-Prinzips kommen Transaktions-Manager und Logging-Mechanismen zum Einsatz.

Die vier ACID-Grundprinzipien

Das ACID-Konzept besteht aus vier einzelnen Grundprinzipien. Diese Grundprinzipien lauten:
\begin{itemize}
	\item Atomicity oder Atomarität: Ausführung aller oder keiner Informationsteile einer Transaktion
	\item Consistency oder Konsistenz: Transaktionen erzeugen einen gültigen Zustand oder fallen in den alten Zustand zurück
	\item Isolation oder Abgrenzung: Transaktionen verschiedener Anwender oder Prozesse bleiben voneinander isoliert
	\item Durability oder Dauerhaftigkeit: Nach einer erfolgreichen Transaktion bleiben die Daten dauerhaft gespeichert
\end{itemize}
\clearpage
