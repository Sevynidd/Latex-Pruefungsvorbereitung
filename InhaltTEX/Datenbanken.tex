\section{Datenbanken}
\label{sec:Datenbanken}

\begin{center}
	\begin{tabular}{|l | p{0.8\textwidth}|}
		\hline
		Datentypen & Boolesche Werte: \texttt{BOOLEAN} \newline Ganzzahl: \texttt{INTEGER} \newline Gleitkommawerte, Währung, Datumswerte, Texte fester und variabler Länge, BLOB, Geokoordinaten \\
		\hline
	\end{tabular}
\end{center}

\begin{itemize}[noitemsep]
	\item OpenData, API-Schnittstellen
	\item Berücksichtigung vorhandener Datenbank- und Speicherkonzepte bei der Integration und Erweiterung von Bestandssystemen
	\item Inbetriebnahme von Speicherlösungen und Integration von Datenbanksystemen
	\item Beachten von Schnittstellen zu weiteren Systemen
	\item Datenquellen: nicht nur relationelle und schemafreie Datenbanken wie MySQL, MsSQL und MongoDB, sondern auch z. B. Sensoren, CSV-Dateien
	\item Begriffe kennen und erläutern
	\begin{itemize}[noitemsep]
		\item Redundanz
		\item Kardinalitäten: 1:1, 1:n, m:n
		\item Primär-/Fremdschlüssel und andere Schlüsseltypen: anonym, künstlich/natürlich
		\item referentielle Integrität (Aktualisierungsweitergabe, Löschweitergabe)
		\begin{itemize}[noitemsep]
			\item Maßnahmen bei Löschoperationen (Constraints): CASCADE, DENY/RESTRICT, SET NULL, (NO ACTION)
		\end{itemize}
		\item Tiefergehende Datenbankobjekte: Index, Stored Procedure, Trigger, Sequence
		\item Replikation
	\end{itemize}
	\item ACID-Prinzipien für Transaktionen kennen und erläutern (atomicity, consistency, isolation, durability)
\end{itemize}

\subsection{Datenbankmodelle und -modellierung}
\label{sec:Datenbankmodelle}

\subsubsection{Relationale und nicht-relationale Datenbanken}
\label{sec:RelationaleDatenbanken}

Eine nicht relationale Datenbank ist eine Datenbank, die nicht das tabellarische Schema mit Zeilen und Spalten verwendet, das in den meisten herkömmlichen Datenbanksystemen zum Einsatz kommt. Nicht relationale Daten verwenden stattdessen ein Speichermodell, das für die spezifischen Anforderungen des gespeicherten Datentyps optimiert ist. So können die Daten beispielsweise als einfache Schlüssel-Wert-Paare, als JSON-Dokumente oder als Diagramm mit Edges und Scheitelpunkten gespeichert werden. \cite{microsoftNichtRelational}

\subsubsection{ERD (Entity-Relationship-Modell)}
\label{sec:ERD}

\begin{minipage}{0.20\textwidth}
	\begin{tikzpicture}[align=center, node distance=1cm]
		\node (entity) [entity] {Entität};
		\node (relationship) [relationship, below=of entity] {Beziehung};
		\node (attribute) [attribute, below=of relationship] {Attribut};
	\end{tikzpicture}
\end{minipage}
\hfill
\begin{minipage}{0.70\textwidth}
	Kardinalitäten:\\
\end{minipage}



\subsection{Normalisierung}
\label{sec:Normalisierung}

Quelle: DatabaseCamp \cite{normalisierung}

\paragraph{1. Normalform} Eine Relation liegt in der ersten Normalform vor, wenn alle Attributwerte \hl{atomar} vorliegen.

Das bedeutet, dass jedes Datenfeld lediglich einen Wert enthalten darf. Außerdem sollte sichergestellt sein, dass jede Spalte nur Werte desselben Datentyps (Numerisch, Text, etc.) enthält. Folgende Beispiele müssten entsprechend verändert werden, damit eine Datenbank in der 1. Normalform vorhanden ist:

\begin{multicols}{2}
	\begin{itemize}
		\item Adresse: “Hauptstraße 1, 12345 Berlin” 
		\begin{itemize}
			\item Straße: “Hauptstraße” 
			\item Hausnummer: “1”
			\item PLZ: “12345”
			\item Ort: “Berlin”
		\end{itemize}
		\columnbreak
		\item Rechnungsbetrag: “128,45 €”
		\begin{itemize}
			\item Betrag: “128,45”
			\item Währung: “€”
			\vfill
		\end{itemize}
	\end{itemize}
\end{multicols}

\paragraph{2. Normalform} Eine Relation liegt in der zweiten Normalform vor, wenn sie in der ersten Normalform vorliegt und alle \hl{Nichtschlüsselattribute voll funktional vom gesamten Primärschlüssel abhängig} sind.

Der Primärschlüssel bezeichnet ein Attribut, das zur eindeutigen Identifikation einer Datenbankzeile verwendet werden kann. Dazu zählen beispielsweise die Rechnungsnummer zur Identifikation einer Rechnung oder die Ausweisnummer zur Identifikation einer Person.

Konkret bedeutet dies in der Anwendung, dass alle Merkmale ausgelagert werden müssen, die nicht ausschließlich vom Primärschlüssel abhängig sind. In der Praxis führt dies dann oft zu einem sogenannten Sternschema.



\paragraph{3. Normalform} Eine Relation liegt in der dritten Normalform vor, wenn sie in der ersten und zweiten Normalform vorliegt und \hl{keine transitiven Abhängigkeiten} bestehen.

\subsubsection{Anomalien}

\paragraph{Einfügeanomalie}

\paragraph{Änderungsanomalie}

\paragraph{Löschanomalie}

\subsection{SQL}
\label{sec:SQL}

Alle SQL-Komponenten für Abfragen siehe \referenz{sec:Datenbanken}.

\subsubsection{Projektion vs. Selektion}
\label{sec:ProjektionSelektion}

Quelle: Tino Hempel \cite{projektionSelektion}

\paragraph{Selektion} Bei der Selektion werden \hl{Zeilen aus einer Tabelle} ausgewählt, die bestimmten Eigenschaften genügen.

Aus der Tabelle Schüler sollen alle Zeilen selektiert werden, in denen der Name "Müller" steht. 
Die Selektion hat also die Form: $S_{Name} = _{'Mueller'}(Schueler)$

\vspace{1em}

\begin{minipage}{.45\textwidth}
	\begin{center}
		Schüler \\
		\vspace{1em}
		\bgroup
		\setlength{\tabcolsep}{1em}
		\def\arraystretch{1.5}
		\begin{tabular}{|c|l|l|}
			\hline
			\rowcolor{tableLightGray}\underline{SNr} & Vorname & Name \\
			\hline
			4711 & Paul & Müller \\
			\hline
			0815 & Erich & Schmidt \\
			\hline
			7472 & Sven & Lehmann \\
			\hline
			1234 & Olaf & Müller \\
			\hline
			2313 & Jürgen & Paulsen \\
			\hline
		\end{tabular}
		\egroup
	\end{center}
\end{minipage}
\hfill
\begin{minipage}{.45\textwidth}
	\begin{center}
		$S_{Name} = _{'Mueller'}(Schueler)$ \\
		\vspace{1em}
		\bgroup
		\setlength{\tabcolsep}{1em}
		\def\arraystretch{1.5}
		\begin{tabular}{|c|l|l|}
			\hline
			\rowcolor{tableLightGray}\underline{SNr} & Vorname & Name \\
			\hline
			12 & Paul & Müller \\
			\hline
			308 & Olaf & Müller \\
			\hline
		\end{tabular}
		\egroup
	\end{center}
\end{minipage}

\paragraph{Projektion} Bei der Projektion werden \hl{Spalten aus einer Tabelle} ausgewählt, die bestimmten Eigenschaften genügen.

Aus der Tabelle Schüler sollen alle Spalten mit dem Attribut 'Name' projiziert werden. 
Die Projektion hat also die Form: $P_{Name}(Schueler)$

\vspace{1em}

\begin{minipage}{.45\textwidth}
	\begin{center}
		Schüler \\
		\vspace{1em}
		\bgroup
		\setlength{\tabcolsep}{1em}
		\def\arraystretch{1.5}
		\begin{tabular}{|c|l|l|}
			\hline
			\rowcolor{tableLightGray}\underline{SNr} & Vorname & Name \\
			\hline
			4711 & Paul & Müller \\
			\hline
			0815 & Erich & Schmidt \\
			\hline
			7472 & Sven & Lehmann \\
			\hline
			1234 & Olaf & Müller \\
			\hline
			2313 & Jürgen & Paulsen \\
			\hline
		\end{tabular}
		\egroup
	\end{center}
\end{minipage}
\hfill
\begin{minipage}{.45\textwidth}
	\begin{center}
		$P_{Name}(Schueler)$ \\
		\vspace{1em}
		\bgroup
		\setlength{\tabcolsep}{1em}
		\def\arraystretch{1.5}
		\begin{tabular}{|l|}
			\hline
			\rowcolor{tableLightGray} Name \\
			\hline
			Müller \\
			\hline
			Schmidt \\
			\hline
			Lehmann \\
			\hline
			Müller \\
			\hline
			Paulsen \\
			\hline
		\end{tabular}
		\egroup
	\end{center}
\end{minipage}


\subsubsection{DDL, DML \& DCL}
\label{sec:DDLDMLDCL}

Quelle: GeeksforGeeks \cite{DDLDMLDCL}

\begin{enumerate}
	\item DDL - Data Definition Language
	\item DML - Data Manipulation Language
	\item DQL - Data Query Language
	\item DCL - Data Control Language
\end{enumerate}

\begin{center}
	\begin{tikzpicture}[
	item/.style={rectangle,draw,fill=blue!30, inner sep = 10},
	subitem/.style={rectangle,draw,fill=blue!20, inner sep = 8},
	grandchild/.style={grow=down,xshift=1em,anchor=west,
		edge from parent path={(\tikzparentnode.south) |- (\tikzchildnode.west)}},
	first/.style={level distance=6ex},
	second/.style={level distance=12ex},
	third/.style={level distance=18ex},
	forth/.style={level distance=24ex},
	level 1/.style={sibling distance=10em}]
	% Parents
	\coordinate
	node[item,anchor=east]{SQL Commands}
	child[grow=down,level distance=2.5ex]
	[edge from parent fork down]
	% Children and grandchildren
	child{node[item] {DDL}
		child[grandchild,first] {node[subitem]{CREATE}}
		child[grandchild,second] {node[subitem]{DROP}}
		child[grandchild,third] {node[subitem] {ALTER}}
		child[grandchild, forth] {node[subitem] {TRUNCATE}}}
	child{node[item] {DML}
		child[grandchild,first] {node[subitem]{INSERT}}
		child[grandchild,second] {node[subitem]{UPDATE}}
		child[grandchild,third] {node[subitem]{DELETE}}}
	child {node[item]{DQL}
		child[grandchild,first] {node[subitem]{SELECT}}}
	child {node[item]{DCL}
		child[grandchild,first] {node[subitem]{GRANT}}
		child[grandchild,second] {node[subitem]{REVOKE}}};
\end{tikzpicture}
\end{center}

\subsubsection{CRUD}
\label{sec:CRUD}

Quelle: sqlshack.com \cite{crud}

\paragraph{C} refers to CREATE aka add, insert. In this operation, it is expected to insert a new record using the SQL insert statement. SQL uses \hl{INSERT INTO} statement to create new records within the table.

\begin{lstlisting}[language=SQL]
	INSERT INTO <tablename> (column1,column2,...) 
	VALUES(value1,value2,...),( value1,value2,...), (value1,value2,...)...
\end{lstlisting}

\begin{lstlisting}[language=SQL]
	INSERT INTO dbo.Demo
	(id, name)
	VALUES
	(2, 'Jayaram'),
	(3, 'Pravitha');
\end{lstlisting}

\paragraph{R} refers to SELECT (data retrieval) operation. The word ‘read’ retrieves data or record-set from a listed table(s). SQL uses the SELECT command to retrieve the data.

\begin{lstlisting}[language=SQL]
	SELECT * FROM <TableName>;
\end{lstlisting}

\paragraph{U} refers to Update operation. Using the Update keyword, SQL brings a change to an existing record(s) of the table. When performing an update, you’ll need to define the target table and the columns that need to update along with the associated values, and you may also need to know which rows need to be updated. In general, you want to limit the number of rows in order to avoid lock escalation and concurrency issues.

\begin{lstlisting}[language=SQL]
	UPDATE <TableName>
	SET Column1=Value1, Column2=Value2,...
	WHERE <Expression>
\end{lstlisting}

\paragraph{D} refers to removing a record from a table. SQL uses the SQL DELETE command to delete the record(s) from the table.

\begin{lstlisting}[language=SQL]
	DELETE FROM <TableName>
	WHERE <Expression>
\end{lstlisting}

\subsubsection{Subqueries}
\label{sec:Subquery}



\subsubsection{Aggregatfunktionen}
\label{sec:Aggregatfunktionen}

\subsubsection{Schnitt-, Vereinigungs- und Differenzmenge}
\label{sec:SchnittVereinigungsDifferenzmenge}

\subsubsection{SQL Injection}
\label{sec:SQLInjection}

