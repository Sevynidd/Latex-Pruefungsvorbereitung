\subsection{Rechnerarchitektur}
\label{sec:Rechnerarchitektur}

\paragraph{CPU}

Die CPU (Central Processing Unit) ist das Herzstück eines Computers und ist für die Ausführung von Programmen und die Verarbeitung von Daten verantwortlich. Sie führt die Befehle aus, die von der Software geladen werden, und koordiniert die Operationen anderer Hardwarekomponenten. Die CPU besteht aus verschiedenen Einheiten, darunter die Rechen- und Steuerungseinheiten, sowie Register und Caches, die zur temporären Speicherung von Daten und Befehlen verwendet werden.

\paragraph{BUS}

Der Bus ist ein System aus elektrischen Leitungen oder Verbindungen, das die Kommunikation zwischen den verschiedenen Komponenten eines Computers ermöglicht. Er dient als Datenübertragungsweg für die Übertragung von Daten, Befehlen und Steuersignalen zwischen der CPU, dem Arbeitsspeicher, den Peripheriegeräten und anderen Komponenten des Computers. Der Bus kann intern, um die Kommunikation innerhalb des Computers zu ermöglichen, oder extern, um die Kommunikation mit externen Geräten wie Festplatten und Druckern zu ermöglichen, sein.

\paragraph{Speicher und deren Adressierung}

Der Speicher eines Computers besteht aus verschiedenen Typen von Speichermedien, darunter der Hauptspeicher (RAM), der Cache-Speicher und der Massenspeicher (Festplatte, SSD). Der Hauptspeicher dient zur temporären Speicherung von Daten und Programmen, die während der Ausführung vom Prozessor benötigt werden. Die Adressierung des Speichers erfolgt über ein Adressbus-System, das es der CPU ermöglicht, auf bestimmte Speicheradressen zuzugreifen und Daten zu lesen oder zu schreiben. Die Größe des Adressbusses bestimmt die maximale Speicherkapazität, die von der CPU adressiert werden kann, während die Breite des Datenbusses die Anzahl der Datenbits angibt, die gleichzeitig übertragen werden können.