\subsection{Allgemeines Fehlerhandling bei Programmen}
\label{sec:Fehlerhandling}

\paragraph{Exception}

Exceptions sind Ereignisse, die während der Programmausführung auftreten und den normalen Ablauf unterbrechen. Sie signalisieren, dass ein Fehler oder eine unerwartete Situation aufgetreten ist, die nicht von der aktuellen Programmausführung behandelt werden kann. Exceptions können verschiedene Ursachen haben, wie z.B. ungültige Eingaben, nicht verfügbare Ressourcen oder interne Fehler. In vielen Programmiersprachen können Exceptions mit try-catch-Blöcken behandelt werden, wodurch der Programmfluss kontrolliert und Fehlerbehandlungsroutinen ausgeführt werden können.

\paragraph{Return/Exit Codes}

Return- oder Exit-Codes sind numerische Werte, die von einem Programm zurückgegeben werden, um den Status der Programmausführung anzuzeigen. Sie werden häufig verwendet, um anzuzeigen, ob ein Programm erfolgreich ausgeführt wurde oder ob ein Fehler aufgetreten ist. Üblicherweise wird ein Wert ungleich Null verwendet, um anzuzeigen, dass ein Fehler aufgetreten ist, während der Wert Null normalerweise für eine erfolgreiche Ausführung steht. Diese Codes können von aufrufenden Programmen oder Skripten verwendet werden, um auf den Erfolg oder das Scheitern der Ausführung zu reagieren und entsprechende Maßnahmen zu ergreifen.