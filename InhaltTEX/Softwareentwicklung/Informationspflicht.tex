\subsection{Informationspflicht}
\label{sec:Informationspflicht}

\paragraph{Namensrecht}

Das Namensrecht regelt die rechtlichen Aspekte der Verwendung von Namen von Personen oder Unternehmen. Es schützt das Recht einer Person oder Firma, ihren eigenen Namen zu verwenden und verbietet die unbefugte Nutzung oder Verwendung des Namens einer anderen Person oder Firma, insbesondere wenn dadurch Verwechslungen oder Irreführungen entstehen könnten.

\paragraph{Markenrecht}

Das Markenrecht regelt den Schutz von Marken und Kennzeichen, die verwendet werden, um Produkte oder Dienstleistungen eines Unternehmens von denen anderer zu unterscheiden. Es gewährt einem Unternehmen das ausschließliche Recht, seine Marke zu verwenden und verbietet anderen die unbefugte Verwendung oder Nachahmung dieser Marke, insbesondere wenn dadurch Verwechslungen oder Irreführungen entstehen könnten.

\paragraph{Urheberrecht}

Das Urheberrecht schützt die geistigen Eigentumsrechte an kreativen Werken wie Literatur, Kunst, Musik, Filmen und Software. Es gewährt dem Urheber das ausschließliche Recht, sein Werk zu reproduzieren, zu verbreiten, öffentlich aufzuführen oder anzuzeigen, abgeleitete Werke zu erstellen und darüber zu entscheiden, wie das Werk verwendet wird. Das Urheberrecht entsteht automatisch, sobald ein Werk in einer fixierten Form erstellt wird, und erfordert keine formale Registrierung.

\paragraph{Nutzungsrecht}

Das Nutzungsrecht ist das Recht, ein urheberrechtlich geschütztes Werk zu nutzen oder zu verwenden, das dem Inhaber des Urheberrechts gewährt wird. Es erlaubt Dritten die Verwendung des Werks unter bestimmten Bedingungen, die in einer Lizenzvereinbarung festgelegt sind. Diese Bedingungen können die Art und Weise der Nutzung, die Dauer der Nutzung, die Gebühren oder Lizenzgebühren sowie andere Einschränkungen oder Anforderungen umfassen.

\paragraph{Persönlichkeitsrecht}

Das Persönlichkeitsrecht schützt die persönlichen und individuellen Rechte einer Person, insbesondere das Recht auf Privatsphäre, das Recht auf informationelle Selbstbestimmung und das Recht am eigenen Bild. Es verbietet die unerlaubte Veröffentlichung oder Verbreitung persönlicher Informationen oder Abbildungen einer Person, insbesondere wenn dadurch die Persönlichkeitsrechte verletzt werden könnten.

\paragraph{Unlauterer Wettbewerb}

Der unlautere Wettbewerb bezieht sich auf rechtswidrige Praktiken oder Verhaltensweisen von Unternehmen, die darauf abzielen, den Wettbewerb zu verzerren oder andere Marktteilnehmer zu benachteiligen. Dazu gehören beispielsweise irreführende Werbung, Verleumdung, Rufschädigung, Preisabsprachen, Behinderung des Marktzugangs oder Verletzungen von geistigen Eigentumsrechten. Der unlautere Wettbewerb wird durch Gesetze und Vorschriften geregelt, die darauf abzielen, fairen und offenen Wettbewerb zu fördern und den Schutz von Verbrauchern und Unternehmen zu gewährleisten.