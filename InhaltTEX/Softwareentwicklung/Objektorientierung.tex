\subsection{Objektorientierung}
\label{sec:Objektorientierung}

In der Objektorientierung geht es darum, die reale Welt im Programm abzubilden. Um dies zu erreichen, werden verschiedene Konzepte gebündelt, die man zusammen unter der \textbf{objektorientierten Programmierung} kennt.

\subsubsection{Objekte und Klassen}

Um Informationen zu speichern und zu manipulieren, benötigen wir etwas, was mit Daten arbeiten kann. Etwas, was Daten halten und verändern kann.

\paragraph{Klasse}

Eine Klasse ist ein Bauplan, in dem definiert wird, was unsere Entität \textbf{ist} und was sie \textbf{kann}.

Hierfür sprechen wir von \textbf{Attributen} und \textbf{Methoden}.

Attribute sind Container, die Daten über und / oder für die Entität speichern.
Methoden hingegen sind vordefinierte Programmabläufe, die eine Entität ausführen kann, die Daten manipulieren.

\newpage

\begin{mdframed}
	Beispiel: Als Klasse könnte man \textbf{Auto} definieren. Dieses Auto besitzt die Attribute:
	\begin{itemize}
		\item Marke
		\item Baujahr
		\item Motortyp
	\end{itemize}
	und die Methoden:
	\begin{itemize}
		\item fahren()
		\item tanken()
	\end{itemize}
\end{mdframed}

\paragraph{Objekt}

Objekte sind die zur Laufzeit \textbf{instanziierten} Klassen. Das bedeutet, dass auf Basis der vorher definierten Klasse ein Objekt erstellt wird. Die Attribute des Objekts macht es einzigartig. Das Objekt kann die Informationen in seinen Attributen verwenden und ihre Methoden ausführen, um Daten zu manipulieren.

\subsubsection{Vererbung}
\textbf{1. Vererbung}\\

\begin{itemize}
	\item Prinzipien der OOP
	\begin{itemize}
		\item Begriffe der OOP erläutern: Attribut, Nachricht/Methodenaufruf, Persistenz, Schnittstelle/API/Interface, Polymorphie, Vererbung
		\item Bestandteile von Klassen
		\item Unterschied Klasse/Objekt
		\item Unterschied Klasse/Interface
		\item Erklärung Klassenbibliothek vs. Framework
		\item Klassenbeziehungen: Assoziation, Aggregation, Komposition, Spezialisierung, Generalisierung
	\end{itemize}
	\item Unterschied statische/nicht-statische Methoden und Attribute
	\item Datenstrukturen (Baum, Array)
	\item funktionale Aspekte in modernen Sprachen: Lambda-Ausdrücke, Functional Interfaces, Map/Filter/Reduce, deklarativ vs. imperativ
\end{itemize}