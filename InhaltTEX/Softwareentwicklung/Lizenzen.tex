\subsection{Lizenzen}
\label{sec:Lizenzen}

\paragraph{Open Source}

Open-Source-Lizenzen sind Lizenzvereinbarungen, die es Entwicklern erlauben, den Quellcode einer Software frei einzusehen, zu modifizieren und weiterzuverbreiten. Sie fördern die Zusammenarbeit, Transparenz und den freien Austausch von Software und ermöglichen es Entwicklern, auf bereits existierende Codebasen aufzubauen und diese weiterzuentwickeln. Beispiele für Open-Source-Lizenzen sind die GPL, Apache License, MIT License und BSD License.

\paragraph{Proprietär}

Proprietäre Lizenzen sind Lizenzvereinbarungen, die es den Rechteinhabern erlauben, die Verwendung, Verbreitung und Modifikation ihrer Software einzuschränken und zu kontrollieren. Sie bieten in der Regel eingeschränkten oder keinen Zugriff auf den Quellcode und erfordern eine Lizenzgebühr oder andere Einschränkungen für die Nutzung der Software. Proprietäre Software wird häufig von Unternehmen entwickelt und vertrieben und umfasst Produkte wie Microsoft Windows, Adobe Photoshop und Oracle Database.

\paragraph{EULA}

Eine Endbenutzer-Lizenzvereinbarung (EULA) ist eine rechtliche Vereinbarung zwischen dem Hersteller oder Anbieter von Software und dem Endbenutzer, die die Bedingungen für die Nutzung der Software festlegt. Sie regelt die Rechte und Pflichten der Benutzer in Bezug auf die Software, einschließlich der Lizenzbedingungen, Nutzungseinschränkungen, Garantieausschlüsse und Haftungsbeschränkungen. Benutzer müssen der EULA zustimmen, bevor sie die Software installieren oder verwenden dürfen.

\paragraph{OEM}

Original Equipment Manufacturer (OEM) bezieht sich auf Softwarelizenzen, die von Hardwareherstellern erworben und auf ihren Geräten vorinstalliert sind. Diese Lizenzen gelten nur für die spezifische Hardware, auf der sie vorinstalliert sind, und sind in der Regel nicht übertragbar. OEM-Lizenzen werden oft zu einem reduzierten Preis angeboten und enthalten möglicherweise Einschränkungen oder zusätzliche Bedingungen im Vergleich zu regulären Einzelhandelslizenzen.

\paragraph{GNU}

Die GNU General Public License (GPL) ist eine weit verbreitete Open-Source-Lizenz, die von der Free Software Foundation (FSF) entwickelt wurde. Sie gewährt Benutzern das Recht, die Software frei zu verwenden, zu modifizieren und weiterzuverbreiten, solange sie die Bedingungen der Lizenz einhalten. Die GPL ist eine sogenannte Copyleft-Lizenz, die sicherstellt, dass abgeleitete Werke unter denselben Lizenzbedingungen veröffentlicht werden müssen.

\paragraph{Apache 2.0}

Die Apache License 2.0 ist eine Open-Source-Lizenz, die von der Apache Software Foundation entwickelt wurde. Sie ermöglicht es Benutzern, die Software frei zu verwenden, zu modifizieren und weiterzuverbreiten, sowohl für kommerzielle als auch für nichtkommerzielle Zwecke. Die Apache License 2.0 ist eine permissive Lizenz, die Benutzern mehr Freiheiten gewährt als die GPL, da sie keine Anforderungen für die Veröffentlichung des Quellcodes für abgeleitete Werke enthält.

\paragraph{GPL 3.0}

Die GNU General Public License (GPL) Version 3.0 ist eine aktualisierte Version der GPL, die von der Free Software Foundation veröffentlicht wurde. Sie enthält aktualisierte Bestimmungen, um mit neuen Entwicklungen in der Softwarebranche Schritt zu halten, einschließlich der Behandlung von Patenten, digitalen Rechteverwaltungen und Webanwendungen. Die GPL 3.0 ist eine Copyleft-Lizenz, die sicherstellt, dass abgeleitete Werke unter denselben Lizenzbedingungen veröffentlicht werden müssen, und sie enthält zusätzliche Bestimmungen zum Schutz der Freiheiten der Benutzer.