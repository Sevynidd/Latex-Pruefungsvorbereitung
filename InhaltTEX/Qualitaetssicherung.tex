\section{Qualitätssicherung}
\label{sec:Qualitätssicherung}

\subsection{PDCA-Zyklus}
\label{sec:PDCA-Zyklus}

Quelle: der-prozessmanager.de \cite{pdcaZyklus}

Der PDCA-Zyklus (auch Deming-Kreis, Deming-Zyklus oder PDCA Kreislauf) bezeichnet ein grundlegendes Konzept im kontinuierlichen Verbesserungsprozess. Es dient der Weiterentwicklung von Produkten und Dienstleistungen sowie bei der Fehler-Ursache-Analyse. Der PDCA-Kreis besteht aus den vier sich wiederholenden Phasen: \hl{Plan-Do-Check-Act} (dt. Planen – Umsetzen – Überprüfen – Handeln). 

\begin{center}
	\includegraphics[scale=0.3]{Bilder/PDCA.png}
\end{center}

\textbf{Vorteile des PDCA-Kreises}

Der wesentliche Vorteil der PDCA-Methode ist wohl die \hl{einfache Anwendbarkeit}. Das Vorgehen hinsichtlich der spezifischen Aufgaben und Problemstellungen kann nahezu uneingeschränkt angepasst werden. Mit den Schritten Plan, Do, Check und Act bleibt dennoch ein solides Gerüst bestehen.

Weitere Vorteile:

\begin{itemize}
	\item Benötigt wenig Anleitung auf Grund des einfachen Aufbaus
	\item Die kreisförmige Konzeption ermöglicht ständige Verbesserung
	\item Durch den iterativen Ansatz lässt der PDCA-Zyklus Kontrolle und Analyse zu
\end{itemize}

\textbf{Nachteile des PDCA-Kreises}

Der große Vorteil des Demingkreises ist gleichzeitig auch ein wesentlicher Nachteil: \hl{Schnelle Problemlösungen lassen sich mit Hilfe des PDCA-Zyklus \mbox{\underline{nicht}} umsetzen}.

Nachteile auf einen Blick:

\begin{itemize}
	\item Unklare Definition der einzelnen Schritte kann zu falschem Einsatz führen
	\item Verbesserungen im Unternehmen müssen langfristig gedacht sein
	\item Eher ein reaktiver Ansatz, statt proaktiv
\end{itemize}

\subsection{Incident Management}
\label{sec:IncidentManagement}

Quelle: it-processmaps.com \cite{incidentManagement}

Incident Management verwaltet alle Incidents über ihren gesamten Lebenszyklus. Das primäre Ziel dieses ITIL-Prozesses besteht darin, einen IT Service für den Anwender so schnell wie möglich wieder herzustellen.

ITIL unterscheidet zwischen \hl{Incidents} (Service-Unterbrechungen) und \hl{Service Requests} (d. h. Anfragen von Anwendern, die keine Service-Unterbrechungen betreffen, wie z.B. das Zurücksetzen eines Passworts). Service-Unterbrechungen werden vom Incident-Management-Prozess behandelt, während Service-Anfragen vom \hl{Request Fulfilment} bearbeitet werden.

Der Incident-Management-Prozess kann auf verschiedenen Wegen angestoßen werden: Ein Anwender, Kunde oder Supplier kann eine Störung melden, technisches Personal kann einen (drohenden oder tatsächlichen) Ausfall feststellen, oder ein Incident kann automatisch von einem Event-Monitoring-System ausgelöst werden.

Alle \hl{Incidents sollten in Incident Records festgehalten werden}, so dass ihr Status verfolgt und ihr vollständiger Verlauf dokumentiert werden kann. Die initiale Kategorisierung und Priorisierung der Incidents ist ein wichtiger Schritt zur Bestimmung, wie mit dem Incident verfahren wird und wieviel Zeit für dessen Lösung verfügbar ist.

Falls möglich, sollten Incidents mit anderen Incidents, Problems und Known Errors verknüpft werden.

\subsection{Service Level Agreement (SLA), Servicelevel 1-3}
\label{sec:ServiceLevelAgreement}

Quelle: Amazon \cite{SLAAmazon}

Ein Service Level Agreement (SLA) ist ein Vertrag mit einem Outsourcing- und Technologieanbieter, in dem das \hl{Serviceniveau} festgelegt ist, das ein Anbieter dem Kunden zu liefern verspricht. Sie gibt \hl{Aufschluss über Kennzahlen wie Betriebszeit, Lieferzeit, Reaktionszeit und Lösungszeit}. In einem SLA ist auch festgelegt, was zu tun ist, wenn die Anforderungen nicht erfüllt werden, z. B. zusätzliche Unterstützung oder Preisnachlässe. SLAs werden in der Regel zwischen einem Kunden und einem Service-Anbieter vereinbart, obwohl auch Geschäftseinheiten innerhalb desselben Unternehmens untereinander SLAs abschließen können.

\subsection{Testen}
\label{sec:Testen}

\clearpage

\subsection{Versionsverwaltung}
\label{sec:Versionsverwaltung} 

Quelle: Atlassian \cite{gitCommands}

\textbf{git commands}

\begin{center}
	\setlength\arrayrulewidth{1pt}
	\begin{tabular}{|p{0.2\textwidth} | p{0.7\textwidth}|}
		\hline
		\textbf{git add} & Verschiebt Änderungen aus dem Arbeitsverzeichnis in die Staging-Umgebung. Auf diese Weise kannst du einen Snapshot vorbereiten, bevor du an den offiziellen Verlauf committest.\\
		\hline
		\textbf{git branch} & Dieser Befehl ist dein Allzwecktool zur Branch-Administration. Damit kannst du isolierte Entwicklungsumgebungen innerhalb eines einzigen Repositorys erstellen.\\
		\hline
		\textbf{git checkout} & Neben dem Auschecken alter Commits und alter Dateiüberarbeitungen kannst du mit 'git checkout' auch zwischen bestehenden Branches navigieren. In Kombination mit den grundlegenden Git-Befehlen kann dadurch in einer bestimmten Entwicklungslinie gearbeitet werden.\\
		\hline
		\textbf{git clone} & Erstellt eine Kopie eines bestehenden Git-Repositorys. Klonen ist für Entwickler die gängigste Art, eine Arbeitskopie eines zentralen Repositorys zu erhalten.\\
		\hline
		\textbf{Git commit} & Committet den Snapshot aus der Staging-Umgebung in den Projektverlauf. Zusammen mit 'git add' bildet er den grundlegenden Workflow für alle Git-Benutzer.\\
		\hline
		\textbf{git fetch} & Mit 'git fetch' wird ein Branch von einem anderen Repository zusammen mit allen zugehörigen Commits und Dateien heruntergeladen. Dabei wird jedoch nichts in dein lokales Repository integriert. Auf diese Weise hast du die Möglichkeit, Änderungen vor dem Merge in dein Projekt noch zu überprüfen.\\
		\hline
		\textbf{git init} & Initialisiert ein neues Git-Repository. Wenn du für ein Projekt eine Versionskontrolle einrichten möchtest, ist dies der erste Befehl, den du kennen musst.\\
		\hline
		\textbf{git log} & Damit kannst du ältere Überarbeitungen eines Projekts ansehen. Der Befehl bietet mehrere Formatierungsoptionen zur Anzeige committeter Snapshots.\\
		\hline
		\textbf{git merge} & Eine leistungsstarke Option zur Integration von Änderungen von voneinander abweichenden Branches. Nach dem Forken des Projektverlaufs mit 'git branch', kann diese mit 'git merge' wieder zusammengeführt werden.\\
		\hline
		\textbf{git pull} & Pulls sind die automatisierte Version von git fetch. Dabei wird ein Branch von einem Remote-Repository heruntergeladen und dann direkt in den aktuellen Branch gemergt. Dies ist das Git-Äquivalent von svn update.\\
		\hline
	\end{tabular}
\end{center}

\begin{center}
	\setlength\arrayrulewidth{1pt}
	\begin{tabular}{|p{0.2\textwidth} | p{0.7\textwidth}|}
		\hline
		\textbf{git push} & 'git push' ist das Gegenteil von 'git fetch' (mit ein paar Einschränkungen). Du kannst mit diesem Befehl einen lokalen Branch in ein anderes Repository verschieben, was eine bequeme Methode zur Veröffentlichung von Beiträgen ist. Dies ist wie 'svn commit', aber hierbei wird eine Reihe von Commits statt eines einzigen Changesets gesendet.\\
		\hline
		\textbf{git rebase} & Mit Rebasing kannst du Branches verschieben, um unnötige Merge-Commits zu vermeiden. Der daraus resultierende lineare Verlauf ist oft leichter zu verstehen und zu durchsuchen.\\
		\hline
		\textbf{git revert} & Macht einen committeten Snapshot rückgängig. Wenn du einen fehlerhaften Commit entdeckst, kannst du ihn mit 'git revert' sicher und einfach von der Codebasis entfernen.\\
		\hline
		\textbf{git status} & Gibt den Status des Arbeitsverzeichnisses und den Status des Snapshots in der Staging-Umgebung zurück. Diesen Befehl solltest du zusammen mit 'git add' und 'git commit' ausführen, um genau zu sehen, was im nächsten Snapshot enthalten sein wird.\\
		\hline
	\end{tabular}
\end{center}

