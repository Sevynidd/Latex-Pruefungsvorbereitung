\section{IT-Sicherheit}
\label{sec:IT-Sicherheit}

\subsection{Datensicherheit}
\label{sec:Datensicherheit}

\subsubsection{Vertraulichkeit, Integrität, Verfügbarkeit (C.I.A Prinzip)}
\label{sec:VertraulichkeitIntegritaetVerfuegbarkeit}

\quelle{Enginsight}{ciaPrinzip}

\paragraph{Vertraulichkeit} Ein System liefert Vertraulichkeit, wenn niemand unautorisiert Informationen gewinnen kann.

\paragraph{Integrität} Ein System gewährleistet die Integrität, wenn es nicht möglich ist, zu schützende Daten unautorisiert und unbemerkt zu verändern.

\paragraph{Verfügbarkeit} Ein System gewährt Verfügbarkeit, wenn authentifizierte und autorisierte Subjekte in der Wahrnehmung ihrer Berechtigungen nicht unautorisiert beeinträchtigt werden können.

\clearpage

\subsubsection{USV (Unterbrechungsfreie Stromversorgung)}
\label{sec:USV}

\quelle{hagel-it}{USV}

\textbf{Was ist eine USV?}

Viele Geräte, wie Server und Router müssen hoch verfügbar sein, um z.B. das Internet am Laufen zu halten. Nicht nur Hard- und Software Probleme, sondern auch die Stromversorgung bildet eine Sicherheitslücke. Daher werden meist sensible IT-Systeme mit einer USV (unterbrechungsfreie Stromversorgung) ausgestattet. Denn im Falle eines Netzausfalls möchte vermieden werden, dass beispielsweise dem Server plötzlich der Strom fehlt und ausgeschaltet wird. Viel mehr möchte man erreichen, dass der Server in diesem Fall sauber und ordentlich herunterfährt.

\subsubsection{Firewall}
\label{sec:Firewall}

\quelle{cisco}{Firewall}

Eine Firewall ist eine Netzwerksicherheitsvorrichtung, die eingehenden und ausgehenden Netzwerkverkehr überwacht und auf Grundlage einer Reihe von definierten Sicherheitsregeln entscheidet, ob bestimmter Datenverkehr zugelassen oder blockiert wird.

Firewalls bilden bereits seit über 25 Jahren die erste Verteidigungslinie beim Schutz von Netzwerken. Sie fungieren als Barriere zwischen geschützten und kontrollierten Bereichen des internen, vertrauenswürdigen Netzwerks und nicht vertrauenswürdigen, äußeren Netzwerken wie dem Internet. 

Eine Firewall kann Hardware, Software, Software-as-a-Service (SaaS), eine Public Cloud oder eine Private Cloud (virtuell) sein.

\clearpage

\subsubsection{Schutzbedarfskategorien}
\label{sec:Schutzbedarf}

Quelle: BSI \cite{BSISchutzbedarf}

\paragraph{Beispiel} Für das Beispielunternehmen, die RECPLAST GmbH, wurde bezüglich der Schadensszenarien „finanzielle Auswirkungen“ und „Beeinträchtigung der Aufgabenerfüllung“ folgendes festgelegt:

{\sethlcolor{green}\hl{Normaler Schutzbedarf:}}

\begin{itemize}
	\item „Der mögliche finanzielle Schaden ist kleiner als 50.000 Euro.“
	\item „Die Abläufe bei RECPLAST werden allenfalls unerheblich beeinträchtigt. Ausfallzeiten von mehr als 24 Stunden können hingenommen werden.“
\end{itemize}

{\sethlcolor{yellow}\hl{Hoher Schutzbedarf:}}

\begin{itemize}
	\item „Der mögliche finanzielle Schaden liegt zwischen 50.000 und 500.000 Euro.“
	\item „Die Abläufe bei RECPLAST werden erheblich beeinträchtigt. Ausfallzeiten dürfen maximal 24 Stunden betragen.“
\end{itemize}

{\sethlcolor{red}\hl{Sehr hoher Schutzbedarf:}}

\begin{itemize}
	\item „Der mögliche finanzielle Schaden liegt über 500.000 Euro.“
	\item „Die Abläufe bei RECPLAST werden so stark beeinträchtigt, dass Ausfallzeiten, die über zwei Stunden hinausgehen, nicht toleriert werden können.“
\end{itemize}

\subsubsection{Begriffe zu Hacking}
\label{sec:Hackingbegriffe}

\quelle{Buch zu Hacking von mitp}{BuchHacking}

\subsubsection{Hacker}
\label{sec:Hacker}

\paragraph{Scriptkiddies} Sie haben wenig Grundwissen und versuchen, mithilfe von Tools in fremde Systeme einzudringen. Dabei sind diese Tools meist sehr einfach über eine Oberfläche zu bedienen. Die Motivation ist meistens Spaß und die Absichten sind oft krimineller Natur. Oftmals möchten Scriptkiddies mit ihren Aktionen Unruhe stiften. Die Angriffe sind meist ohne System und Strategie. Viele Hacker starten ihre Karriere als Scriptkiddie, nutzen die Tools zunächst mit wenig Erfahrung, lernen aus dem Probieren, entwickeln sich weiter und finden dadurch einen Einstieg in die Szene.

\paragraph{Black Hats} Diese Gattung Hacker beschreibt am ehesten die Hacker, die man aus den Medien kennt. Hier redet man von Hackern mit bösen Absichten. Sie haben sehr gute Kenntnisse und greifen bewusst und strukturiert Unternehmen, Organisationen oder Einzelpersonen an, um diesen Schaden zuzufügen. Die Ziele der Black Hats sind vielfältig und reichen vom einfachen Zerstören von Daten bis hin zum Diebstahl von wertvollen Informationen, wie Kontodaten oder Unternehmensgeheimnissen. In manchen Fälle reicht es den Black Hats auch, wenn sie erfolgreich die Server ihres Opfers lahmlegen und damit Sabotage verüben.

\paragraph{White Hats} Einen \textit{White Hat Hacker} nennt man oft auch einen \textit{Ethical Hacker}. Er nutzt das Wissen und die Tools eines Hackers, um zu verstehen, wie Black Hats bei ihren Angriffen vorgehen. Im Gegensatz zum Black Hat will der White Hat jedoch die betreffenden Systeme letztlich vor Angriffen besser schützen und testet daher die Schwachstellen aktiv aus. Damit hat ein White Hat Hacker grundsätzlich keine bösen Absichten, im Gegenteil, er unterstützt die Security-Verantwortlichen der jeweiligen Organisation. White Hat Hacker oder Ethical Hacker versuchen im Anschluss an ihre Hacking-Tätigkeit herauszufinden, welche Sicherheitslücken es gibt und geben eine Anleitung dazu, diese möglichst effizient zu schließen.

\subsubsection{Phishing}
\label{sec:Phishing}

Beim Phishing versucht der Phisher in der Regel, an sensible bzw. persönliche Daten des Opfers zu gelangen. Dies sind meistens entweder Login-Informationen oder Kreditkartendaten - im einfachsten Fall einfach eine valide E-Mail-Adresse. Dies wird durch gefakte E-Mails, Webseiten oder Kurznachrichten erreicht.

\subsubsection{Spoofing}
\label{sec:Spoofing}

\todo{Fehlt}

\subsubsection{Session Hijacking}
\label{sec:SessionHijacking}

Mit >>Session Hijacking<< ist die Übernahme einer Session zwischen zwei Systemen durch einen Angreifer gemeint. Session Hijacking ist eine perfide Angriffsform, da sich der Angreifer sozusagen >>ins gemachte Nest<< setzt. Dabei können diverse Schwachstellen auf unterschiedlichen Ebenen der Kommunikation ausgenutzt werden.


\subsubsection{Man-in-the-Middle (MITM)}
\label{sec:ManInTheMiddle}

Bei einer MITM-Attacke platziert sich der Angreifer so, dass er die relevante Kommunikation zwischen Person 1 und Person 2 über ein vomn ihm kontrolliertes System leitet und somit den gesamten Traffic mitlesen und ggf. sogar manipulieren kann. Dies geschieht für Person 1 und Person 2 komplett transparent. Das bedeutet, dass Person 1 glaubt, mit Person 2 direkt zu kommunizieren, und umgekehrt. Tatsächlich aber gibt sich der Angreifer gegenüber Person 1 als Person 2 aus und gegenüber Person 2 tut er so, als wäre er Person 1.


\subsubsection{SQL-Injection}
\referenz{sec:SQLInjection}

\subsubsection{DoS, DDoS}
\label{sec:DoSDDoS}

{
\vspace{1em}
\bgroup
\setlength{\tabcolsep}{1em}
\def\arraystretch{1.5}
\begin{tabular}{l l}
	\textbf{DoS} & \textit{Denial-of-Service} \\
	\textbf{DDos} & \textit{Distributed-Denial-of-Service}
\end{tabular}
\egroup
}

\vspace{1em}

Unter einem \textit{Denial-of-Service-Angriff} (DoS) verstehen wir jede Art von Angriff, die das Ziel verfolgt, die Verfügbarkeit eines Computersystems oder Netzwerkdienstes zu reduzieren oder komplett zu verhindern. Diese Nichtverfügbarkeit kann auf ganz verschiedenen Wegen erreicht werden. Letztlich geht es aber immer darum, dass eine Komponente ihren Dienst nicht so verrichten kann wie vorgesehen, da deren Funktionalität willentlich und vorsätzlich angegriffen wurde.

Dabei wird das Opfer-System in der Regel mit Dienstanfragen oder Datenpaketen überhäuft, um dessen Ressourcen aufzubrauchen und eine Überlastung zu erreichen. Das Ziel eines DoS-Angriffs ist rein destruktiv.

In den meisten Fällen basieren DoS-Angriffe schlicht auf einer massiven Überlastung der Opfer-Systeme. Nach dem Motto >>Viel hilft viel<< werden heutzutage häufig konzentrierte Angriffe von Hunderten oder Tausenden Systemen aus dem Internet durchgeführt. Dies nennen wir \textit{Distributed-Denial-of-Service} (DDoS-Attacke).

\subsubsection{Viren}
\label{sec:Viren}

Der oder das Computervirus (beides ist mittlerweile statthaft) ist klassischerweise ein Programm, das sich selbst verbreitet, indem es sich in ein anderes programm, den Wirt, einschleust. Im Gegensatz zu Trojanern oder Würmern benötigen Viren also einen Wirt und können sich nicht selbstständig fortpflanzen.

In dem Moment, in dem das Wirtsprogramm ausgeführt wird, kann auch der Virus aktiv werden und seine Payload ausführen. Zur Replikation führt der Virus immer einen Prozess aus, der es ihm ermöglicht, sich an einen neuen Wirt zu hängen, um sich weiterzuverbreiten.

\subsubsection{Würmer}
\label{sec:Wuermer}

Würmer können sich - im Gegensatz zu Viren - selbstständig und ohne Wirt vermehren. Sie verbreiten sich über Netzwerke oder Wechselmedien, wie USB-Sticks. Dadurch können sie sich höchst effektiv replizieren und verbreiten.

Ebenso wie Trojaner und Viren können auch Würmer einen Schadcode ausführen.

\subsubsection{Trojaner}
\label{sec:Trojaner}

Die Trojaner werden als harmloses Programm getrarnt, enthalten aber Schadcode, der im Hintergrund ohne Wissen des Anwenders ausgeführt wird. Trojaner sind die flexibelste und die am weitesten verbreitete Form von Malware. Im Grunde genommen ist ein Trojaner nur eine Tarnung für eine beliebige weitere Malware-Funktion wie z.B. Installation von:

\begin{multicols}{2}
	\begin{itemize}
		\item Backdoors
		\item Keylogger
		\item Ransomware
		\item Sniffer
		\item Bots für ein Botnet
		\item und so weiter
	\end{itemize}
\end{multicols}

\subsubsection{Sniffer, Spyware und Keylogger}
\label{sec:SnifferSpywareKeylogger}

Zahlreiche Schadprogramme versuchen, den Anwender auszuspionieren, und senden die Daten anschließend an vordefinierte Zielsysteme im Internet. Die Spionagetätigkeiten können diverse Aspekte und Komponenten umfassen. Angefangen von Daten wie Browser-Verlauf und anderen Zugriffsverläufen (\textit{Most Recently Used}, MRU) über das Mitschneiden des Netzwerk-Traffics oder Aktivieren von Mikrofon und Webcam bis hin zum Protokollieren jedes einzelnen Tastenanschlags mittels Keylogger kann ein Angreifer umfassende Daten über das Opfer sammeln.

\subsubsection{Botnetze}
\label{sec:Botnetze}

Ein \textit{Botnet} besteht aus zahlreichen \textit{Bots}, also einem Stückchen Software, das sich im Opfer-System eingenistet hat und bereit ist, vom Command \& Control-Server aus dem Internet Befehle zu empfangen. Bots könenn diverse Funktionen erfüllen - angefangen vom Spamversand über Spyware-Funktionen bis hin zu gezielten und konzentrierten \textit{Distributed-Denial-of-Service-Angriffen}.

\subsubsection{Scareware}
\label{sec:Scareware}

Eine perfide Art, Malware auf ein Opfer-System zu bringen, ist die \textit{Scareware}. Hierbei täuscht ein Popup einer Website oder eine als kostenloses Antiviren-Programm verteilte Software dem Benutzer den Fund zahlreicher, gefährlicher Vireninfektionen vor, die über ein zu installierendes, ggf. kostenpflichtiges Programm beseitigt werden können.


\subsubsection{Kryptotrojaner \& Ransomware}
\label{sec:Ransomware}

Kryptografie soll die Sicherheit erhöhen. Dank ausgeklügelter, komplexer mathematischer Verfahren und Algorithmen gelingt dies in der Regel sehr effektiv und für die >>bösen Jungs<< ist es teilweise sehr schwierig bis unmöglich, kryptografisch gesicherte Daten und Datenströme zu knacken bzw. zu entschlüsseln. Umso perfider wird es, wenn diese Technologien gegen uns verwendet werden. So geschehen bei den sogenannten \textit{Kryptotrojanern} bzw. der gefürchteten \textit{Ransomware}.

Hinter Kryptotrojanern steckt sehr viel kriminelle Energie, denn die Angreifer wollen in fast allen Fällen Geld ergaunern. Die Schädlinge verschlüsseln persönliche Daten auf dem System des Opfers so, dass sie für den Eigentümer unbrauchbar sind. Es gibt auch Varianten, die den Zugriff auf das komplette System blockieren. Erst gegen Bezahlung eines Lösegeldes (engl. \textit{Ransom}) werden die Dateien entschlüsselt und damit wieder verfügbar gemacht. Nachdem das Opfer über eine anonyme Zahlungsmethode, wie zum Beispiel die Kryptowährung \textit{Bitcoin}, das Lösegeld bezahlt hat, ist natürlich nicht gewährleistet, dass die Daten tatsächlich wieder freigegeben werden!

\subsubsection{Backdoor}
\label{sec:Backdoor}

Eine >>Backdoor<<, oder deutsch: >>Hintertür<<, bezeichnet ganz allgemein einen Zugang zu einem System unter Umgehung der Zugriffsschutzmaßnahmen. Der Zweck einer Backdoor besteht in der Regel darin, einem nicht authentisierten und autorisierten Benutzer einen jederzeit verfügbaren Zugriff zum Zielsystem bereitzustellen. Dies impliziert, dass dieser Zugang den autorisierten Benutzern meistens nicht bekannt ist bzw. sein soll.


\subsubsection{Exploit}
\label{sec:Exploit}

Ein \textit{Exploit} ist ein Prozess, mit dem Sie eine Schwachstelle ausnutzen können. Doch was bedeutet >>ausnutzen<< eigentlich? Das hängt ganz von der Schwachstelle und vom Exploit ab. In unserem Fall geht es darum, eine \textit{Remote-Shell} bereitzustellen - egal ob Bind- oder Reverse-Shell. Diese Shell ist die \textit{Payload} des Exploits. Durch die Schwachstelle wird z.B. die Ausführung beliebiger Kommandos möglich. Der Exploit setzt dies technisch um. Die Payload stellt dann genau diese Befehle dar, mit denen der Angreifer Zugriff auf das System erlangt.

\subsubsection{Rootkit}
\label{sec:Rootkit}

Eine sehr effektive Möglichkeit, Malware und unerwünschte Prozesse zu verstecken und zu tarnen, besteht darin, sie für den Anwender und das System unsichtbar zu machen. Der Begriff >>Rootkit<< setzt sich aus dem Linux/UNIX-Administator \textit{root} und dem Wort >>Kit<< zusammen und bedeutet wörtlich übersetzt ungefähr >>Administrator-Werkzeugkasten<<. Damit wird zum Ausdruck gebracht, dass es sich eigentlich um eine Toolsammlung handelt.

\subsubsection{Verbreitung von Viren/Würmern/Trojanern}
\label{sec:VerbreitungVirenWuermerTrojaner}

\begin{itemize}
	\item \textit{Websites:} Eines der wichtigsten Einfallstore für Malware sind kompromittierte Websites. Besucht das Opfer die Website, gibt es verschiedene Möglichkeiten, die der Angreifer ausnutzen kann.
	\begin{itemize}
		\item Via \textit{Drive-by-Downloads} wird Software ohne Wissen und Zutun des Benutzers automatisch heruntergeladen und installiert. Hierzu werden Browser-Schwachstellen ausgenutzt, da regulär HTML-Code oder Browser-Skriptsprachen kein Zugriff auf Bereiche außerhalb der Browser-Umgebung gestattet wird. Die Voraussetzung hierfür ist also ein ungepatchter Browser oder eine Zero-Day-Schwachstelle.
		\item Noch direkter ist die Bereitstellung von Trojanern im Rahmen der Software-Angebote der Website. So gelang es Angreifern zum Beispiel, die beliebte und äußerst nützliche Software \textit{CCleaner} in der Version 5.33 mit Malware zu verseuchen, sodass aus dem Programm ein Trojaner wurde.
	\end{itemize}
	\item \textit{Instant Messenger:} Fast jeder Messenger bietet die Möglichkeit der Dateiübertragung. Wird das Opfer dazu verführt, ein Programm entgegenzunehmen und zu installieren, hat der Angreifer sein Ziel erreicht.
	\item \textit{Wechselmedien (USB-Sticks):} Ein Klassiker ist auch die Verteilung von Malware über USB-Sticks, die das Opfer anschließt. Mittlerweile ist diese Gefahr etwas entschäft, da die Autorun-Funktion bei neueren Windows-Versionen standardmäßig deaktiviert ist. Dennoch bleibt das Risiko, dass der Benutzer eine spannend klingende Programmdatei ausführt und sich damit Malware auf den Computer lädt.
	\item \textit{E-Mail-Anhänge:} Ebenfalls eine der klassischen Methoden, um Malware zu verbreiten. Insbesondere Makroviren über Word- und Excel-Dateien lassen sich über diesen Weg hervorragend verteilen. Auch diverse weitere Dateitypen sind gefährdet und möglicherweise infizierbar. Seit einiger Zeit werden viele manipulierte PDF-Dateien mit angeblichen Lebensläufen verschickt. Die E-Mails sind als Blindbewerbung getarnt.
	\item \textit{Links in E-Mails und Websites:} Manchmal ist es nicht notwendig, die Malware direkt zu übergeben. Auch Hyperlinks dienen dazu, den ahnungslosen Benutzer auf kompromittierte Websites zu leiten oder direkt einen Download zu starten.
	\item \textit{File Sharing:} Ein beliebter Weg zur Verteilung von Malware besteht im Austausch von Dateien und Programmen via File Sharing. Fake-Programme und diverse andere trojanerbasierende Tools mit vielversprechenden Inhalten und Funktionen verleiten zum Download. Das betrifft z.B. Spiele, Bildschirmschoner, >>Crackz und Warez<< und so weiter.
	\item \textit{Windows-Freigaben:} Die Windows-Netzwerkumgebung bietet einfachen Zugriff auf freigegebene Ordner anderer Windows-Computer im Netzwerk. Gelingt es dem Angreifer, hier zentrale Programmdateien durch infizierte Varianten auszutauschen, kann er unter Umständen mit einem Zug diverse Systeme infizieren, wenn das betreffende Programm regelmäßig von Benutzern auf dem lokalen Computer ausgeführt wird.
	\item \textit{Software-Bugs:} Enthält eine Software Schwachstellen - allen voran Browser und E-Mail-Clients -, kann die Malware auch über diesen Weg auf den Computer des Opfers gelangen. Die oben erwähnten Drive-by-Downloads basieren auf Schwachstellen im Browser, wie bereits geschrieben. Aber auch andere programme, wie z.B. Flash-Player oder PDF-Reader, können für derartige Angriffe anfällig sein.
	\item \textit{Fehlkonfiguration:} Ein sehr wichtiger Punkt ist die Konfiguration einer Software. Unter bestimmten Bedingungen ist eine Software entweder per Default oder durch den Administrator unsicher konfiguriert und damit anfällig für entsprechende Angriffe, die die Installation von Malware nach sich ziehen.
\end{itemize}

