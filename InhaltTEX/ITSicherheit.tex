\section{IT-Sicherheit}
\label{sec:IT-Sicherheit}

\begin{itemize}
	\item Datensicherheit (Authentifizierung, Autorisierung, Verschlüsselung)
	\item Bedrohungsszenarien erkennen und Schadenspotenziale unter Berücksichtigung wirtschaftlicher und technischer Kriterien einschätzen
	\item Für jede Anwendung, die verwendeten IT-Systeme und die verarbeiteten Informationen gilt: Betrachtung zu erwartender Schäden, die bei einer Beeinträchtigung von Vertraulichkeit, Integrität oder Verfügbarkeit entstehen könnten
	\item geeignete Gegenmaßnahmen, z.B. USV-Anlagen, Klimageräte, Firewalls
	\item Einteilung in die drei Schutzbedarfskategorien „normal“, „hoch“ und „sehr hoch“ (analog IT-Grundschutz des BSI)
	\item Begriffe kennen/erläutern
	\begin{itemize}
		\item Hacker (White Hat, Black Hat), Cracker, Script-Kiddies
		\item Spam, Phishing, Sniffing, Spoofing, Man-in-the-Middle
		\item SQL-Injection, XSS, CSRF, Session Hijacking, DoS, DDoS
		\item Viren, Würmer, Trojaner, Hoax, Dialer (veraltet), Keylogger, Botnetze, Spyware, Adware, Ransomware, Scareware
		\item Backdoor, Exploit, 0-Day-Exploit, Rootkit
		\item Verbreitung von Viren/Würmer/Trojaner erläutern
	\end{itemize}
	
\end{itemize}