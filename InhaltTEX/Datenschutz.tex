\section{Datenschutz}
\label{sec:Datenschutz}

\todo{Fehlt}


\paragraph{DSGVO} Datenschutzgrundverordnung

\paragraph{BDSG} Bundesdatenschutzgesetz


\begin{center}
	\begin{tikzpicture}[
	node distance=1cm, 
	align=center,
	every node/.style={minimum height=1.5cm, minimum width=10cm, draw}]
	
	\node (BDSG){BDSG};
	\node (DSGVO)[below=of BDSG]{DSGVO};
	\draw [->, thick] (DSGVO) -- (BDSG);
	
\end{tikzpicture}
\end{center}

\quelle{b-quadrat}{unterschiedDSGVOBDSG}

Ein entscheidender Unterschied zwischen DSGVO und BDSG liegt in ihrem Geltungsbereich. Die DSGVO ist eine Verordnung der Europäischen Union und gilt daher unmittelbar in allen Mitgliedsstaaten. Sie betrifft Unternehmen, die personenbezogene Daten von EU-Bürgern verarbeiten, unabhängig von ihrem Sitz. Dadurch wird ein einheitlicher Datenschutzstandard in der gesamten EU gewährleistet.

Das BDSG, dagegen, ist spezifisch auf Deutschland ausgerichtet. Es ergänzt die DSGVO um nationale Bestimmungen und legt besondere Anforderungen an die Datenverarbeitung in Deutschland fest. Es berücksichtigt dabei kulturelle und rechtliche Besonderheiten, die in Deutschland gelten.




\begin{itemize}
	\item Datenschutzgesetze – national und auf EU-Ebene, z.B. Datenschutzgrundverordnung (DSGVO), BDSG
	\item Grundsätze des Datenschutzes (Art. 5)
	\begin{itemize}
		\item Rechtmäßigkeit/Gesetzmässigkeit (Erfordernis der gesetzlichen Grundlage)
		\item Transparenz gegenüber den betroffenen Personen
		Zweckbindung
		\item Datenminimierung/Verhältnismässigkeit (Datensparsamkeit und Datenvermeidung)
		\item Richtigkeit
		\item Speicherbegrenzung
		\item Integrität und Vertraulichkeit
		\item Rechenschaftspflicht
		\item Informationssicherheit
	\end{itemize}
	\item Betroffenenrechte
	\begin{itemize}
		\item Recht auf Information
		\item Recht auf Auskunft
		\item Recht auf Berichtigung
		\item Recht auf Löschung
		\item Recht auf Einschränkung der Bearbeitung
		\item Recht auf Widerspruch
		\item Recht auf Datenübertragbarkeit
	\end{itemize}
	\item Persönlichkeitsrechte
	\begin{itemize}
		\item Recht auf informationelle Selbstbestimmung
		\item Recht am eigenen Bild
		\item Recht am geschriebenen/gesprochenen Wort
		\item Recht auf Schutz vor Imitation der Persönlichkeit
		\item Recht auf Schutz der Intim-, Privat- und Geheimsphäre
	\end{itemize}
\end{itemize}