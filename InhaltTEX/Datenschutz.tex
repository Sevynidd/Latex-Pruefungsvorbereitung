\section{Datenschutz}
\label{sec:Datenschutz}

\paragraph{DSGVO} Datenschutzgrundverordnung

\paragraph{BDSG} Bundesdatenschutzgesetz


\begin{center}
	\begin{tikzpicture}[
	node distance=1cm, 
	align=center,
	every node/.style={minimum height=1.5cm, minimum width=10cm, draw}]
	
	\node (BDSG){BDSG};
	\node (DSGVO)[below=of BDSG]{DSGVO};
	\draw [->, thick] (DSGVO) -- (BDSG);
	
\end{tikzpicture}
\end{center}

\quelle{b-quadrat}{unterschiedDSGVOBDSG}

Ein entscheidender Unterschied zwischen DSGVO und BDSG liegt in ihrem Geltungsbereich. Die DSGVO ist eine Verordnung der Europäischen Union und gilt daher unmittelbar in allen Mitgliedsstaaten. Sie betrifft Unternehmen, die personenbezogene Daten von EU-Bürgern verarbeiten, unabhängig von ihrem Sitz. Dadurch wird ein einheitlicher Datenschutzstandard in der gesamten EU gewährleistet.

Das BDSG, dagegen, ist spezifisch auf Deutschland ausgerichtet. Es ergänzt die DSGVO um nationale Bestimmungen und legt besondere Anforderungen an die Datenverarbeitung in Deutschland fest. Es berücksichtigt dabei kulturelle und rechtliche Besonderheiten, die in Deutschland gelten.

\subsection{Grundsätze des Datenschutzes}
\label{sec:GrundsaetzeDesDatenschutzes}

\quelle{IHK Rhein-Neckar}{grundsaeteDatenschutz}

\paragraph{Rechtmäßigkeit} Die Verarbeitung personenbezogener Daten ist rechtmäßig, wenn eine Einwilligung des Betroffenen (Art. 6 Abs. 1 S. 1 lit. a) DSGVO) oder eine gesetzliche Erlaubnis (Art. 6 Abs. 1 S. 1 lit. b) bis f) DSGVO) vorliegt.

\paragraph{Transparenz} Es muss für den Betroffenen immer ersichtlich sein, für welchen Zweck seine personenbezogenen Daten verarbeitet werden. Eine „heimliche“ Verarbeitung ist unzulässig.

\paragraph{Zweckbindung} Bei der Verarbeitung personenbezogener Daten muss der Zweck der Verarbeitung vor der Erhebung der Daten festgelegt werden. Stellen Sie sich also immer die Frage, wofür konkret Sie die von Ihnen erhobenen Daten verarbeiten wollen. Eine nachträgliche Änderung des Zwecks ist grundsätzlich nicht zulässig. Sie müssen also stets gewährleisten, dass keine zweckentfremdete Verarbeitung der von Ihnen erhobenen Daten stattfindet.

Ferner ist der Zweck maßgebend für die Speicherdauer. Wenn er entfällt, sind Sie verpflichtet, die personenbezogenen Daten zu löschen (Ausnahme: gesetzliche Aufbewahrungspflichten, siehe unter Speicherbegrenzung).

\paragraph{Datenminimierung/-sparsamkeit} Bei der Datenverarbeitung dürfen nur so viele personenbezogene Daten gesammelt werden, wie für den jeweiligen Verarbeitungszweck unbedingt notwendig sind. Es gilt der Grundsatz: „So viele Daten wie nötig, so wenige Daten wie möglich.“ Dadurch soll der Betroffene vor einer übermäßigen Preisgabe personenbezogener Daten geschützt werden.

\paragraph{Richtigkeit} Personenbezogene Daten müssen richtig und aktuell sein sowie aus zuverlässigen Quellen stammen. Unrichtige oder veraltete Daten müssen unmittelbar gelöscht oder korrigiert werden.

\paragraph{Speicherbegrenzung (Löschung/Sperrung)} Werden personenbezogene Daten nicht mehr benötigt, müssen sie gelöscht werden, es sei denn, der Löschung stehen gesetzliche Aufbewahrungspflichten (insbesondere im Handels- und Steuerrecht) entgegen. Solange die Aufbewahrungsfrist läuft, werden die Daten zwar nicht gelöscht, aber für eine weitere Nutzung durch den Verantwortlichen gesperrt.

\paragraph{Integrität und Vertraulichkeit} Personenbezogene Daten müssen sicher und vertraulich behandelt werden. Insbesondere dürfen Unbefugte keinen Zugang zu ihnen haben und weder die Daten noch die Geräte, mit denen diese verarbeitet werden, benutzen können.

\paragraph{Rechenschaftspflicht (Dokumentation)} Ihr Unternehmen muss gegenüber Aufsichtsbehörden nachweisen können, alle Vorgaben der DSGVO einzuhalten. Aus diesem Grund müssen Sie die von Ihnen getroffenen rechtlichen, technischen und organisatorische Maßnahmen zur Sicherstellung des Datenschutzes genauestens dokumentieren. Dokumentation heißt, dass Sie entsprechende Dokumente, Belege und sonstige Materialien in schriftlicher oder elektronischer Form systematisch aufbewahren und archivieren, damit Sie diese im Ernstfall unverzüglich griffbereit haben. Zu diesen Dokumentationspflichten gehört beispielsweise auch das Führen eines Verarbeitungsverzeichnisses gemäß Art. 30 DSGVO.

\subsection{Betroffenenrechte}
\label{sec:Betroffenenrechte}

\quelle{BfDI}{Betroffenenrechte}

\paragraph{Das Recht auf Auskunft (Art. 15 DSGVO)} Mit dem Auskunftsrecht garantiert Ihnen Art. 15 der Datenschutz-Grundverordnung (DSGVO) ein bedeutsames Betroffenenrecht. Danach können Sie als betroffene Person von dem für die Datenverarbeitung Verantwortlichen Auskunft darüber verlangen, welche Daten dort über Sie gespeichert sind bzw. verarbeitet werden.

\paragraph{Das Recht auf Berichtigung (Art. 16 DSGVO)} Die DSGVO gewährt allen Bürgern, deren personenbezogene Daten verarbeitet werden, eine Vielzahl an Rechten, wie das Recht auf Berichtigung.

\paragraph{Das Recht auf Löschung / 'Recht auf Vergessenwerden' (Art. 17 DSGVO)} Nach der DSGVO haben Sie das Recht auf Löschung Ihrer personenbezogenen Daten sowie auf 'Vergessenwerden'.

\paragraph{Das Recht auf Einschränkung der Verarbeitung (Art. 18 DSGVO)} Ein weiteres Betroffenenrecht, das Ihnen die DSGVO an die Hand gibt, ist das Recht auf Einschränkung der Verarbeitung.

\paragraph{Das Recht auf Widerspruch (Art. 21 DSGVO)} Art. 21 Abs. 1 Datenschutz-Grundverordnung ( DSGVO) gewährt Ihnen das Recht, aus Gründen, die sich aus Ihrer besonderen Situation ergeben, ausnahmsweise auch gegen eine an sich rechtmäßige Datenverarbeitung Widerspruch einzulegen. Einer an sich rechtmäßigen Datenverarbeitung zur Direktwerbung können Sie nach Art. 21 Abs. 2 DSGVO sogar ohne jede Begründung widersprechen.

\begin{itemize}
	\item Persönlichkeitsrechte
	\begin{itemize}
		\item Recht auf informationelle Selbstbestimmung
		\item Recht am eigenen Bild
		\item Recht am geschriebenen/gesprochenen Wort
		\item Recht auf Schutz vor Imitation der Persönlichkeit
		\item Recht auf Schutz der Intim-, Privat- und Geheimsphäre
	\end{itemize}
\end{itemize}